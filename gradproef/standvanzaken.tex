\chapter{\IfLanguageName{dutch}{Stand van zaken}{State of the art}}%
\label{ch:stand-van-zaken}

Dit hoofdstuk biedt een overzicht van de technologieën die ten grondslag liggen aan dit project. Elke technologie wordt besproken in termen van functionaliteit en relevantie voor het AI-callcenter.

\section{Twilio: programmeerbare telefonie}

Twilio is een cloud communications platform dat API's aanbiedt voor spraak, SMS en andere communicatiekanalen~\autocite{Twilio2024}. Het platform maakt het mogelijk om telefoongesprekken volledig programmatisch te beheren, wat essentieel is voor de automatisering van een callcenter.

\subsection{Programmable Voice}

De Programmable Voice API stelt ontwikkelaars in staat om:
\begin{itemize}
    \item Inkomende oproepen te ontvangen en te routeren
    \item Uitgaande oproepen te initiëren
    \item Spraakherkenning (Speech-to-Text) toe te passen voor het transcriberen van gesproken input
    \item Text-to-Speech te gebruiken voor het genereren van gesproken antwoorden
    \item Gesprekken op te nemen voor kwaliteitscontrole en logging
\end{itemize}

\subsection{TwiML}

TwiML (Twilio Markup Language)~\autocite{TwilioTwiML2024} is een XML-gebaseerde taal waarmee instructies aan Twilio worden gegeven. De belangrijkste TwiML-verbs voor dit project zijn:

\begin{itemize}
    \item \textbf{<Say>}: Spreekt tekst uit met Text-to-Speech
    \item \textbf{<Gather>}: Verzamelt spraak- of toetsinvoer van de beller~\autocite{TwilioGather2024}
    \item \textbf{<Record>}: Neemt het gesprek op~\autocite{TwilioRecording2024}
    \item \textbf{<Dial>}: Verbindt door naar een ander telefoonnummer
    \item \textbf{<Hangup>}: Beëindigt het gesprek
\end{itemize}

\subsection{Webhooks}

Twilio werkt met webhooks~\autocite{TwilioWebhooks2024}: HTTP callbacks die bij elke fase van het gesprek worden aangeroepen. Belangrijk is de strikte timeout van 15 seconden --- als de server niet binnen deze tijd reageert, verbreekt Twilio de verbinding. Dit vereist zorgvuldig ontwerp van de backend-architectuur.

\section{OpenAI en Function Calling}

OpenAI's GPT-modellen vormen het hart van de AI-assistent~\autocite{OpenAI2024}. Deze Large Language Models (LLM's) zijn getraind op grote hoeveelheden tekst en kunnen menselijke taal begrijpen en genereren.

\subsection{GPT-4o}

Voor dit project wordt het GPT-4o model gebruikt, dat geoptimaliseerd is voor:
\begin{itemize}
    \item Snelle responstijden (essentieel voor real-time conversatie)
    \item Meertalige ondersteuning
    \item Instructie-volgen (het model houdt zich aan de gegeven richtlijnen)
\end{itemize}

\subsection{Function Calling}

Het function calling mechanisme is cruciaal voor de integratie van AI met backend-systemen. Het stelt het model in staat om niet alleen tekst te genereren, maar ook gestructureerde acties uit te voeren:

\begin{enumerate}
    \item \textbf{Definitie}: Functies worden gedefinieerd met parameters in JSON Schema formaat
    \item \textbf{Herkenning}: Het model analyseert de gebruikersinput en bepaalt of een functie nodig is
    \item \textbf{Aanroep}: De backend ontvangt een gestructureerd verzoek met functienaam en parameters
    \item \textbf{Uitvoering}: De backend voert de actie uit (bijv. database-query)
    \item \textbf{Respons}: Het resultaat wordt teruggegeven aan het model
    \item \textbf{Formulering}: Het model formuleert een natuurlijk antwoord voor de gebruiker
\end{enumerate}

Dit mechanisme maakt het mogelijk om de AI-assistent te verbinden met bestaande bedrijfssystemen zonder dat het model directe toegang tot databases nodig heeft.

\section{ASP.NET Core en Azure}

De backend is gebouwd met ASP.NET Core~\autocite{CSharpDocs2024}, het cross-platform webframework van Microsoft. Deze keuze is gemaakt vanwege de integratie met de bestaande TurnUp-infrastructuur, die volledig op .NET is gebouwd.

\subsection{Azure Services}

De applicatie maakt gebruik van verschillende Azure-services~\autocite{AzureKeyVault2024}:

\begin{itemize}
    \item \textbf{Azure App Service}: Hosting van de webapplicatie met automatische schaling en hoge beschikbaarheid
    \item \textbf{Azure Key Vault}: Veilige opslag van gevoelige gegevens zoals API-sleutels voor Twilio en OpenAI
    \item \textbf{Azure SQL Database}: Opslag van gespreksopnames en transcripties
\end{itemize}

\subsection{Deployment}

Voor continuous deployment wordt GitHub Actions~\autocite{GitHubActions2024} gebruikt, waarmee elke push naar de main branch automatisch wordt gedeployed naar Azure.

\section{Ontwikkelomgeving}

\subsection{ngrok voor lokale ontwikkeling}

Twilio webhooks vereisen een publiek bereikbaar endpoint. Tijdens lokale ontwikkeling wordt ngrok~\autocite{Ngrok2024} gebruikt om een beveiligde tunnel te creëren van een publieke URL naar de lokale ontwikkelserver. Dit maakt het mogelijk om:

\begin{itemize}
    \item Webhooks te ontvangen zonder te deployen
    \item Snel te itereren en te debuggen
    \item Real-time logs te bekijken van inkomende requests
\end{itemize}
