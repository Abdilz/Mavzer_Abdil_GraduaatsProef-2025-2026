%%=============================================================================
%% Resultaten
%%=============================================================================

\chapter{Resultaten}%
\label{ch:resultaten}

Dit hoofdstuk presenteert de resultaten van het ontwikkelde AI-callcenter en bespreekt de opgedane leerpunten tijdens het project.

\section{Functionele resultaten}

Het ontwikkelde systeem voldoet aan alle gestelde functionele eisen:

\begin{table}[h]
\centering
\begin{tabular}{lcc}
\toprule
\textbf{Functionaliteit} & \textbf{Geïmplementeerd} & \textbf{Getest} \\
\midrule
Klantgegevens opzoeken & Ja & Ja \\
Afspraak annuleren & Ja & Ja \\
Afspraak verzetten & Ja & Ja \\
Toevoegen aan wachtlijst & Ja & Ja \\
Doorverbinden naar praktijk & Ja & Ja \\
Meertalige ondersteuning (NL/EN/FR) & Ja & Ja \\
Gespreksopname & Ja & Ja \\
Transcriptie & Ja & Ja \\
\bottomrule
\end{tabular}
\caption[Functionele resultaten]{Overzicht van geïmplementeerde en geteste functionaliteiten}
\end{table}

\subsection{Prestaties}

De belangrijkste prestatie-indicatoren van het systeem:

\begin{itemize}
    \item \textbf{Beschikbaarheid}: 24/7 zonder menselijke tussenkomst
    \item \textbf{Gemiddelde gespreksduur}: minder dan 2 minuten voor standaard scenario's
    \item \textbf{Latency}: gemiddeld 2-4 seconden tussen vraag en antwoord
    \item \textbf{Succesratio}: circa 85\% van de gesprekken wordt succesvol afgehandeld
\end{itemize}

\section{Technische bevindingen}

\subsection{Uitdagingen en oplossingen}

Tijdens de ontwikkeling werden verschillende technische uitdagingen overwonnen:

\subsubsection{Twilio Timeout}

\textbf{Probleem}: Twilio hanteert een strikte timeout van 15 seconden voor webhook responses. Bij complexe OpenAI-aanroepen werd deze soms overschreden.

\textbf{Oplossing}: Implementatie van een interne timeout van 14 seconden met fallback naar een standaard ``Kunt u dat herhalen?'' response.

\subsubsection{Spraakherkenning van telefoonnummers}

\textbf{Probleem}: Telefoonnummers worden door spraakherkenning vaak incorrect getranscribeerd (bijvoorbeeld ``nul vier'' in plaats van ``04'').

\textbf{Oplossing}: Uitgebreide normalisatie-logica die verschillende formaten herkent en converteert naar het standaard E.164 formaat.

\subsubsection{Context behoud over meerdere beurten}

\textbf{Probleem}: De AI vergat soms eerder opgezochte informatie wanneer de gebruiker van onderwerp wisselde.

\textbf{Oplossing}: Implementatie van een dynamische context die bij elke beurt aan de prompt wordt toegevoegd.

\subsubsection{Meertalige prompts}

\textbf{Probleem}: De AI wisselde soms ongewild van taal of mengde talen.

\textbf{Oplossing}: Strikte taalinstructies aan het begin van de prompt en dynamische taalspecifieke voorbeeldzinnen.

\section{Leerpunten}

Dit project heeft geleid tot significante persoonlijke en professionele groei:

\subsection{Technische vaardigheden}

\begin{itemize}
    \item \textbf{Webhooks}: Diepgaande ervaring met asynchroon communiceren tussen externe services en de backend. Het begrijpen van callback-patronen en het correct afhandelen van meerdere gelijktijdige requests.
    \item \textbf{AI-integratie}: Praktische ervaring met prompt engineering en OpenAI's function calling mechanisme.
    \item \textbf{Real-time systemen}: Inzicht in de uitdagingen van lage-latency spraakverwerking en timeout-management.
    \item \textbf{Cloud deployment}: Hands-on ervaring met Azure-services, met name Key Vault voor secrets management.
\end{itemize}

\subsection{Professionele vaardigheden}

\begin{itemize}
    \item \textbf{Zelfstandig werken}: Het project vereiste veel eigen onderzoek en probleemoplossend denken.
    \item \textbf{Communicatie met stakeholders}: Regelmatig overleg met TurnUp over requirements.
    \item \textbf{Iteratief ontwikkelen}: Het belang van kleine, testbare incrementen.
\end{itemize}

\section{Vergelijking met doelstellingen}

\begin{table}[h]
\centering
\begin{tabular}{p{5cm}cc}
\toprule
\textbf{Doelstelling} & \textbf{Doel} & \textbf{Bereikt} \\
\midrule
Zonder menselijke tussenkomst & Ja & Ja \\
Gemiddelde gespreksduur & < 3 min & < 2 min \\
24/7 beschikbaarheid & Ja & Ja \\
Latency vraag-antwoord & < 2 sec & 2-4 sec* \\
Meertalige ondersteuning & 3 talen & 3 talen \\
\bottomrule
\end{tabular}
\caption[Doelstellingen]{Vergelijking doelstellingen. *Acceptabel voor gebruikers.}
\end{table}

\section{Beperkingen}

Het huidige systeem kent enkele beperkingen:

\begin{itemize}
    \item \textbf{Complexe vragen}: Vragen buiten het afspraakdomein kunnen niet worden beantwoord
    \item \textbf{Accenten}: Sterke accenten kunnen de spraakherkenning bemoeilijken
    \item \textbf{Achtergrondlawaai}: In lawaaiige omgevingen daalt de kwaliteit
\end{itemize}
