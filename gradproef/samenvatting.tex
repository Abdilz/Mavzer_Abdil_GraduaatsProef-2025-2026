%%=============================================================================
%% Samenvatting
%%=============================================================================

\chapter*{\IfLanguageName{dutch}{Samenvatting}{Abstract}}

Tandartspraktijken besteden dagelijks aanzienlijke tijd aan telefonisch afspraakbeheer. Receptionisten worden overspoeld met repetitieve telefoontjes, terwijl patiënten buiten kantooruren niet terecht kunnen. Dit onderzoek richt zich op de vraag: \textit{Hoe kan een AI-gestuurd callcenter de telefonische dienstverlening automatiseren met behoud van klanttevredenheid?}

Er werd een volledig geautomatiseerd inbound callcenter ontwikkeld met \textbf{Twilio} voor telefonie-infrastructuur en \textbf{OpenAI GPT} voor natuurlijke taalverwerking. Het systeem is gebouwd met ASP.NET Core en gehost op Microsoft Azure. Via webhooks communiceert Twilio met de backend, waarbij spraakherkenning (Speech-to-Text) en spraaksynthese (Text-to-Speech) naadloos zijn geïntegreerd.

Het systeem ondersteunt drie talen (Nederlands, Engels, Frans) en beschikt over vijf kernfuncties via OpenAI's function calling mechanisme: klantgegevens opzoeken, afspraken annuleren, afspraken verzetten, toevoegen aan wachtlijst, en doorverbinden naar de praktijk. De AI-assistent herkent de intentie van de beller en roept automatisch de juiste functie aan via de TurnUp API.

\textbf{Resultaten:} Het systeem is 24/7 beschikbaar zonder menselijke tussenkomst, met een gemiddelde gespreksduur van minder dan twee minuten. Alle gesprekken worden opgenomen en getranscribeerd voor kwaliteitscontrole.

\textbf{Conclusie:} Conversationele AI kan effectief worden ingezet voor klantenservice in de zorgsector. De combinatie van Twilio en OpenAI biedt een schaalbare, meertalige en kostenefficiënte oplossing.

\bigskip

\textbf{GitHub repository:} \url{https://github.com/Abdilz/Mavzer_Abdil_GraduaatsProef-2025-2026.git}

\newpage

%%---------- Engelse samenvatting (Summary) -----------------------------------

\chapter*{Summary}

Dental practices spend considerable time daily handling appointment-related phone calls. Receptionists are overwhelmed with repetitive calls, while patients cannot reach the practice outside office hours. This research addresses the question: \textit{How can an AI-powered call center automate telephone services for dental practices while maintaining customer satisfaction?}

A fully automated inbound call center was developed using \textbf{Twilio} for telephony infrastructure and \textbf{OpenAI GPT} for natural language processing. The system is built with ASP.NET Core and hosted on Microsoft Azure. Twilio communicates with the backend via webhooks, with speech recognition (Speech-to-Text) and speech synthesis (Text-to-Speech) seamlessly integrated.

The system supports three languages (Dutch, English, French) and features five core functionalities through OpenAI's function calling mechanism: looking up customer data, cancelling appointments, rescheduling appointments, adding patients to the waitlist, and transferring calls to the practice. The AI assistant recognizes the caller's intent and automatically invokes the appropriate function via the TurnUp API.

\textbf{Results:} The system is available 24/7 without human intervention, with an average call duration of less than two minutes. All conversations are recorded and transcribed for quality control.

\textbf{Conclusion:} Conversational AI can be effectively deployed for customer service in the healthcare sector. The combination of Twilio and OpenAI provides a scalable, multilingual, and cost-effective solution.

\bigskip

\textbf{GitHub repository:} \url{https://github.com/Abdilz/Mavzer_Abdil_GraduaatsProef-2025-2026.git}
