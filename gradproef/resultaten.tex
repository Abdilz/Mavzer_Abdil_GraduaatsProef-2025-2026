%%=============================================================================
%% Resultaten
%%=============================================================================

\chapter{Resultaten}%
\label{ch:resultaten}

Dit hoofdstuk presenteert de resultaten van het ontwikkelde AI-callcenter en bespreekt de opgedane leerpunten tijdens het project.

\section{Functionele resultaten}

Het ontwikkelde systeem voldoet aan alle gestelde functionele eisen:

\subsection{Kernfunctionaliteiten}

\begin{table}[h]
\centering
\begin{tabular}{lcc}
\toprule
\textbf{Functionaliteit} & \textbf{Geïmplementeerd} & \textbf{Getest} \\
\midrule
Klantgegevens opzoeken & \checkmark & \checkmark \\
Afspraak annuleren & \checkmark & \checkmark \\
Afspraak verzetten & \checkmark & \checkmark \\
Toevoegen aan wachtlijst & \checkmark & \checkmark \\
Doorverbinden naar praktijk & \checkmark & \checkmark \\
Meertalige ondersteuning (NL/EN/FR) & \checkmark & \checkmark \\
Gespreksopname & \checkmark & \checkmark \\
Transcriptie & \checkmark & \checkmark \\
\bottomrule
\end{tabular}
\caption[Functionele resultaten]{Overzicht van geïmplementeerde en geteste functionaliteiten}
\end{table}

\subsection{Prestaties}

De belangrijkste prestatie-indicatoren van het systeem:

\begin{itemize}
    \item \textbf{Beschikbaarheid}: 24/7 zonder menselijke tussenkomst
    \item \textbf{Gemiddelde gespreksduur}: minder dan 2 minuten voor standaard scenario's
    \item \textbf{Latency}: gemiddeld 2-4 seconden tussen vraag en antwoord
    \item \textbf{Succesratio}: circa 85\% van de gesprekken wordt succesvol afgehandeld
\end{itemize}

\section{Technische bevindingen}

\subsection{Uitdagingen en oplossingen}

Tijdens de ontwikkeling werden verschillende technische uitdagingen overwonnen:

\subsubsection{Twilio Timeout}

\textbf{Probleem}: Twilio hanteert een strikte timeout van 15 seconden voor webhook responses. Bij complexe OpenAI-aanroepen werd deze soms overschreden.

\textbf{Oplossing}: Implementatie van een interne timeout van 14 seconden met fallback naar een standaard ``Kunt u dat herhalen?'' response.

\subsubsection{Spraakherkenning van telefoonnummers}

\textbf{Probleem}: Telefoonnummers worden door spraakherkenning vaak incorrect getranscribeerd (bijvoorbeeld ``nul vier'' in plaats van ``04'').

\textbf{Oplossing}: Uitgebreide normalisatie-logica die verschillende formaten herkent en converteert naar het standaard E.164 formaat.

\subsubsection{Context behoud over meerdere beurten}

\textbf{Probleem}: De AI vergat soms eerder opgezochte informatie wanneer de gebruiker van onderwerp wisselde.

\textbf{Oplossing}: Implementatie van een dynamische context die bij elke beurt aan de prompt wordt toegevoegd, met expliciete instructies over wat al bekend is.

\subsubsection{Meertalige prompts}

\textbf{Probleem}: De AI wisselde soms ongewild van taal of mengde talen.

\textbf{Oplossing}: Strikte taalinstructies aan het begin van de prompt en dynamische taalspecifieke voorbeeldzinnen.

\subsection{Webhook-architectuur}

De webhook-gebaseerde architectuur bleek effectief maar heeft beperkingen:

\textbf{Voordelen}:
\begin{itemize}
    \item Eenvoudig te implementeren en debuggen
    \item Goede ondersteuning door Twilio
    \item Stateless backend mogelijk
\end{itemize}

\textbf{Nadelen}:
\begin{itemize}
    \item Elke beurt vereist een aparte HTTP-aanroep
    \item Latency door network round-trips
    \item Twilio timeout beperkt complexiteit van verwerking
\end{itemize}

\textbf{Toekomstige verbetering}: Overgang naar WebSockets zou de latency significant kunnen verlagen en meer complexe real-time interacties mogelijk maken.

\section{Leerpunten}

Dit project heeft geleid tot significante persoonlijke en professionele groei op verschillende vlakken:

\subsection{Technische vaardigheden}

\begin{itemize}
    \item \textbf{Webhooks}: Diepgaande ervaring met asynchroon communiceren tussen externe services en de backend. Het begrijpen van callback-patronen en het correct afhandelen van meerdere gelijktijdige requests.

    \item \textbf{AI-integratie}: Praktische ervaring met prompt engineering en OpenAI's function calling mechanisme. Het leren balanceren tussen instructie-detail en flexibiliteit in AI-prompts.

    \item \textbf{Real-time systemen}: Inzicht in de uitdagingen van lage-latency spraakverwerking. Het belang van timeout-management en graceful degradation.

    \item \textbf{Cloud deployment}: Hands-on ervaring met Azure-services, met name Key Vault voor secrets management en Blob Storage voor media-opslag.
\end{itemize}

\subsection{Professionele vaardigheden}

\begin{itemize}
    \item \textbf{Zelfstandig werken}: Het project vereiste veel eigen onderzoek en probleemoplossend denken. Documentatie lezen, experimenteren, en itereren waren dagelijkse bezigheden.

    \item \textbf{Communicatie met stakeholders}: Regelmatig overleg met TurnUp over requirements en feedback verwerken in het systeem.

    \item \textbf{Iteratief ontwikkelen}: Het belang van kleine, testbare incrementen in plaats van grote releases.
\end{itemize}

\section{Vergelijking met doelstellingen}

\begin{table}[h]
\centering
\begin{tabular}{p{6cm}cc}
\toprule
\textbf{Doelstelling} & \textbf{Doel} & \textbf{Bereikt} \\
\midrule
Gesprekken zonder menselijke tussenkomst & Ja & Ja \\
Gemiddelde gespreksduur & < 3 min & < 2 min \\
24/7 beschikbaarheid & Ja & Ja \\
Latency vraag-antwoord & < 2 sec & 2-4 sec* \\
Meertalige ondersteuning & 3 talen & 3 talen \\
\bottomrule
\end{tabular}
\caption[Doelstellingen]{Vergelijking van doelstellingen met bereikte resultaten. *Latency hoger dan doel maar acceptabel voor gebruikers.}
\end{table}

De latency-doelstelling van minder dan 2 seconden werd niet volledig gehaald. De werkelijke latency van 2-4 seconden is voornamelijk te wijten aan de verwerkingstijd van OpenAI. Dit bleek in de praktijk geen probleem voor de gebruikerservaring, aangezien patiënten gewend zijn aan korte wachttijden tijdens telefoongesprekken.

\section{Beperkingen}

Het huidige systeem kent enkele beperkingen:

\begin{itemize}
    \item \textbf{Complexe vragen}: Vragen die buiten het afspraakdomein vallen kunnen niet worden beantwoord.
    \item \textbf{Accenten en dialecten}: Sterke accenten kunnen de spraakherkenning bemoeilijken.
    \item \textbf{Achtergrondlawaai}: In lawaaiige omgevingen daalt de kwaliteit van de spraakherkenning.
    \item \textbf{Geen proactieve acties}: Het systeem kan momenteel geen uitgaande oproepen doen (bijvoorbeeld herinneringen).
\end{itemize}

\section{Feedback van stakeholders}

TurnUp heeft positief gereageerd op het ontwikkelde systeem. Specifieke feedback:

\begin{quote}
``Het systeem handelt de meeste standaard scenario's goed af. De meertalige ondersteuning is een grote meerwaarde voor onze Belgische klanten. We zien potentieel voor uitrol naar meerdere praktijken.''
\end{quote}

Verbeterpunten die werden aangedragen:
\begin{itemize}
    \item Meer edge cases afhandelen (bijvoorbeeld gecombineerde vragen)
    \item Kortere antwoorden in sommige scenario's
    \item Betere afhandeling van onduidelijke spraak
\end{itemize}
