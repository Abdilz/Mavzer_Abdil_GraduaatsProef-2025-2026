%===============================================================================
% LaTeX sjabloon voor de graduaatsproef Programmeren aan HOGENT
% Meer info op https://github.com/HoGentPRG/latex-hogent-report
%===============================================================================

\documentclass[dutch,dit,thesis]{hogentreport}

\usepackage{lipsum} % For blind text, can be removed after adding actual content

%% Pictures to include in the text can be put in the graphics/ folder
\graphicspath{{../graphics/}}

%% For source code highlighting, requires pygments to be installed
%% Compile with the -shell-escape flag!
%% \usepackage[chapter]{minted}
%% If you compile with the make_thesis.{bat,sh} script, use the following
%% import instead:
\usepackage[chapter,outputdir=../output]{minted}
\usemintedstyle{solarized-light}

%% Formatting for minted environments.
\setminted{%
    autogobble,
    frame=lines,
    breaklines,
    linenos,
    tabsize=4
}

%% Ensure the list of listings is in the table of contents
\renewcommand\listoflistingscaption{%
    \IfLanguageName{dutch}{Lijst van codefragmenten}{List of listings}
}
\renewcommand\listingscaption{%
    \IfLanguageName{dutch}{Codefragment}{Listing}
}
\renewcommand*\listoflistings{%
    \cleardoublepage\phantomsection\addcontentsline{toc}{chapter}{\listoflistingscaption}%
    \listof{listing}{\listoflistingscaption}%
}

% Other packages not already included can be imported here
\usepackage{tikz}
\usetikzlibrary{positioning,shapes.geometric}

%% Kleur voor Twilio logo
\definecolor{twiliored}{RGB}{240, 77, 82}

%%---------- Document metadata -------------------------------------------------
\author{Abdil Mavzer}
\supervisor{Dhr. L. Vervoort}
\cosupervisor{Dhr. J. Decubber (TurnUp)}
\title[Automatisering van patiëntcommunicatie met Twilio en OpenAI]%
    {AI-gestuurd Callcenter voor Tandartspraktijken}
\academicyear{\advance\year by -1 \the\year--\advance\year by 1 \the\year}
\examperiod{1}
\degreesought{\IfLanguageName{dutch}{Graduaat in het Programmeren}{Associate of applied computer science}}
\partialthesis{false} %% To display 'in partial fulfilment'
\institution{TurnUp}

%% Twilio logo op de titelpagina (groter en gecentreerd)
\institutionlogo{%
    \begin{tikzpicture}[scale=2.0]
        % Main circle
        \fill[twiliored] (0,0) circle (1.2cm);
        % Inner white dots pattern (Twilio's signature dots)
        \fill[white] (-0.4, 0.4) circle (0.18cm);
        \fill[white] (0.4, 0.4) circle (0.18cm);
        \fill[white] (-0.4, -0.4) circle (0.18cm);
        \fill[white] (0.4, -0.4) circle (0.18cm);
    \end{tikzpicture}\\[0.5cm]
    {\LARGE\bfseries\color{twiliored} Twilio}%
}

%% Add global exceptions to the hyphenation here
\hyphenation{back-slash OpenAI Twilio}

%% The bibliography (style and settings are  found in hogentthesis.cls)
\addbibresource{gradproef.bib}            %% Bibliography file
\addbibresource{../voorstel/voorstel.bib} %% Bibliography research proposal
\defbibheading{bibempty}{}

%% Wijzig "Bibliografie" naar "Bronnenlijst"
\DefineBibliographyStrings{dutch}{
    bibliography = {Bronnenlijst},
}

%% Prevent empty pages for right-handed chapter starts in twoside mode
\renewcommand{\cleardoublepage}{\clearpage}

\renewcommand{\arraystretch}{1.2}

%% Content starts here.
\begin{document}

%---------- Front matter -------------------------------------------------------

\frontmatter

\hypersetup{pageanchor=false} %% Disable page numbering references
%% Render a Dutch outer title page if the main language is English
\IfLanguageName{english}{%
    %% If necessary, information can be changed here
    \degreesought{Graduaat in het Programmeren}%
    \begin{otherlanguage}{dutch}%
       \maketitle%
    \end{otherlanguage}%
}{}

%% Generates title page content
\maketitle
\hypersetup{pageanchor=true}

%%=============================================================================
%% Voorwoord
%%=============================================================================

\chapter*{\IfLanguageName{dutch}{Woord vooraf}{Preface}}%
\label{ch:voorwoord}

Deze graduaatsproef is het resultaat van een intensieve periode waarin ik de kans kreeg om mijn passie voor programmeren te combineren met innovatieve technologieën zoals kunstmatige intelligentie en spraakherkenning. Het ontwikkelen van een AI-gestuurd callcenter voor tandartspraktijken was niet alleen een technische uitdaging, maar ook een waardevolle leerervaring op professioneel en persoonlijk vlak.

Mijn interesse in dit onderwerp ontstond tijdens mijn stage bij TurnUp, waar ik kennismaakte met de uitdagingen waarmee tandartspraktijken dagelijks geconfronteerd worden op het gebied van telefonische bereikbaarheid. De mogelijkheid om met cutting-edge technologieën zoals OpenAI en Twilio een concrete oplossing te bouwen, sprak mij enorm aan.

Graag wil ik een aantal mensen bedanken die hebben bijgedragen aan de totstandkoming van dit werk.

Allereerst wil ik mijn promotor, dhr. Luc Vervoort, bedanken voor de begeleiding gedurende het hele traject. Zijn feedback en ondersteuning hebben mij geholpen om gestructureerd te werk te gaan en de kwaliteit van mijn werk te verbeteren.

Daarnaast gaat mijn dank uit naar mijn co-promotor, dhr. Jona Decubber van TurnUp, die mij de kans heeft gegeven om aan dit project te werken. Zijn technische expertise en praktische inzichten waren onmisbaar bij het navigeren door de complexiteit van real-time spraakverwerking en AI-integratie.

Tot slot wil ik mijn familie en vrienden bedanken voor hun steun en geduld gedurende mijn studie. Hun aanmoediging heeft mij gemotiveerd om door te zetten, ook wanneer het even tegenzat.

\bigskip

Abdil Mavzer

Gent, januari 2026

%%=============================================================================
%% Samenvatting
%%=============================================================================

%%---------- Nederlandse samenvatting -----------------------------------------

\chapter*{\IfLanguageName{dutch}{Samenvatting}{Abstract}}

Tandartspraktijken besteden dagelijks aanzienlijke tijd aan het telefonisch afhandelen van afspraakgerelateerde vragen. Receptionisten worden overspoeld met repetitieve telefoontjes voor het bekijken, annuleren of verzetten van afspraken, terwijl patiënten vaak moeten wachten of buiten kantooruren niet terecht kunnen. Dit onderzoek richt zich op de vraag: \textit{Hoe kan een AI-gestuurd callcenter de telefonische dienstverlening van tandartspraktijken automatiseren met behoud van klanttevredenheid?}

Voor dit project werd een volledig geautomatiseerd inbound callcenter ontwikkeld dat gebruik maakt van \textbf{Twilio} voor de telefonie-infrastructuur en \textbf{OpenAI GPT} voor natuurlijke taalverwerking. Het systeem is gebouwd met ASP.NET Core en gehost op Microsoft Azure. De architectuur maakt gebruik van webhooks voor real-time communicatie tussen Twilio en de backend, waarbij spraakherkenning (Speech-to-Text) en spraaksynthese (Text-to-Speech) naadloos zijn geïntegreerd.

Het systeem biedt ondersteuning voor drie talen (Nederlands, Engels en Frans) en beschikt over vijf kernfunctionaliteiten via OpenAI's function calling mechanisme: klantgegevens opzoeken, afspraken annuleren, afspraken verzetten, patiënten toevoegen aan de wachtlijst, en doorverbinden naar de praktijk. De AI-assistent herkent de intentie van de beller en roept automatisch de juiste functie aan via de TurnUp API.

De resultaten tonen aan dat het systeem 24/7 beschikbaar is zonder menselijke tussenkomst, met een gemiddelde gespreksduur van minder dan twee minuten. Alle gesprekken worden opgenomen en getranscribeerd voor kwaliteitscontrole. Het systeem integreert naadloos met de bestaande praktijksoftware van TurnUp.

Dit onderzoek concludeert dat conversationele AI effectief kan worden ingezet voor klantenservice in de zorgsector. De combinatie van Twilio en OpenAI biedt een schaalbare, meertalige en kostenefficiënte oplossing die zowel patiënten als praktijkmedewerkers ontlast. Toekomstige uitbreidingen omvatten de overstap naar WebSockets voor snellere real-time communicatie, outbound calling voor afspraakherinneringen, en RAG-integratie voor het beantwoorden van praktijkspecifieke vragen.

\bigskip

\textbf{GitHub repository:} \url{https://github.com/Abdilz/Mavzer_Abdil_GraduaatsProef-2025-2026.git}

\newpage

%%---------- Engelse samenvatting (Summary) -----------------------------------

\chapter*{Summary}

Dental practices spend considerable time daily handling appointment-related phone calls. Receptionists are overwhelmed with repetitive calls for viewing, cancelling, or rescheduling appointments, while patients often have to wait or cannot reach the practice outside office hours. This research addresses the question: \textit{How can an AI-powered call center automate telephone services for dental practices while maintaining customer satisfaction?}

For this project, a fully automated inbound call center was developed using \textbf{Twilio} for telephony infrastructure and \textbf{OpenAI GPT} for natural language processing. The system is built with ASP.NET Core and hosted on Microsoft Azure. The architecture utilizes webhooks for real-time communication between Twilio and the backend, with speech recognition (Speech-to-Text) and speech synthesis (Text-to-Speech) seamlessly integrated.

The system supports three languages (Dutch, English, and French) and features five core functionalities through OpenAI's function calling mechanism: looking up customer data, cancelling appointments, rescheduling appointments, adding patients to the waitlist, and transferring calls to the practice. The AI assistant recognizes the caller's intent and automatically invokes the appropriate function via the TurnUp API.

Results demonstrate that the system is available 24/7 without human intervention, with an average call duration of less than two minutes. All conversations are recorded and transcribed for quality control. The system integrates seamlessly with TurnUp's existing practice management software.

This research concludes that conversational AI can be effectively deployed for customer service in the healthcare sector. The combination of Twilio and OpenAI provides a scalable, multilingual, and cost-effective solution that benefits both patients and practice staff. Future extensions include transitioning to WebSockets for faster real-time communication, outbound calling for appointment reminders, and RAG integration for answering practice-specific questions.

\bigskip

\textbf{GitHub repository:} \url{https://github.com/Abdilz/Mavzer_Abdil_GraduaatsProef-2025-2026.git}


%---------- Inhoud, lijst figuren, ... -----------------------------------------

\tableofcontents

%% Lijsten weggelaten om binnen 10-15 pagina's te blijven
%\listoffigures
%\listoftables
%\listoflistings

%---------- Kern ---------------------------------------------------------------

\mainmatter{}

% De eerste hoofdstukken van een graduaatsproef zijn meestal een inleiding op
% het onderwerp, literatuurstudie en verantwoording methodologie.
% Aarzel niet om een meer beschrijvende titel aan deze hoofdstukken te geven of
% om bijvoorbeeld de inleiding en/of stand van zaken over meerdere hoofdstukken
% te verspreiden!

%%=============================================================================
%% Inleiding
%%=============================================================================

\chapter{\IfLanguageName{dutch}{Inleiding}{Introduction}}%
\label{ch:inleiding}

Tandartspraktijken ontvangen dagelijks tientallen telefoontjes van patiënten voor afspraakgerelateerde vragen. Deze repetitieve taken leggen druk op het administratief personeel en beperken de bereikbaarheid tot kantooruren. TurnUp, een Belgisch softwarebedrijf gespecialiseerd in praktijkbeheersoftware, zag een opportuniteit om deze uitdaging aan te pakken met AI-technologie.

\section{Probleemstelling}%
\label{sec:probleemstelling}

Tandartspraktijken worden geconfronteerd met:
\begin{itemize}
    \item \textbf{Beperkte bereikbaarheid}: Enkel tijdens kantooruren
    \item \textbf{Wachttijden}: Piekmomenten leiden tot frustratie
    \item \textbf{Repetitieve taken}: Standaardvragen die geautomatiseerd kunnen worden
    \item \textbf{Meertaligheid}: NL/FR/EN ondersteuning vereist
\end{itemize}

\section{Onderzoeksvraag}%
\label{sec:onderzoeksvraag}

\textit{Hoe kan een AI-gestuurd callcenter de telefonische dienstverlening van tandartspraktijken automatiseren met behoud van klanttevredenheid?}

Deelvragen:
\begin{enumerate}
    \item Welke technologieën zijn geschikt voor spraakgestuurde AI-assistenten?
    \item Hoe kan OpenAI's function calling worden ingezet voor afspraakbeheer?
    \item Welke architectuur is nodig voor real-time spraakverwerking?
    \item Hoe kan meertalige ondersteuning worden geïmplementeerd?
\end{enumerate}

\section{Onderzoeksdoelstelling}%
\label{sec:onderzoeksdoelstelling}

Het beoogde resultaat is een werkend prototype dat:
\begin{itemize}
    \item Inkomende telefoontjes automatisch beantwoordt via Twilio
    \item Patiënten in natuurlijke taal te woord staat via OpenAI GPT
    \item Ondersteuning biedt voor Nederlands, Engels en Frans
    \item Afspraken kan opzoeken, annuleren, verzetten en patiënten aan de wachtlijst kan toevoegen
    \item Integreert met de TurnUp API
\end{itemize}

\chapter{\IfLanguageName{dutch}{Stand van zaken}{State of the art}}%
\label{ch:stand-van-zaken}

Dit hoofdstuk biedt een overzicht van de technologieën en concepten die ten grondslag liggen aan dit project. We bespreken achtereenvolgens de evolutie van AI in klantenservice, de werking van de gebruikte platformen (Twilio en OpenAI), en de architecturale overwegingen voor real-time spraakverwerking.

\section{Kunstmatige intelligentie in klantenservice}

De inzet van AI voor klantenservice is de afgelopen jaren exponentieel gegroeid. Waar vroeger eenvoudige Interactive Voice Response (IVR) systemen met vaste menu's de norm waren, maken moderne systemen gebruik van Natural Language Processing (NLP) om natuurlijke gesprekken te voeren~\autocite{OpenAI2023}.

\subsection{Van IVR naar conversationele AI}

Traditionele IVR-systemen werken met vooraf gedefinieerde keuzemenu's: ``Druk 1 voor afspraken, druk 2 voor facturatie.'' Deze aanpak heeft significante beperkingen:

\begin{itemize}
    \item Gebruikers moeten door lange menu's navigeren
    \item Complexe vragen kunnen niet worden afgehandeld
    \item De gebruikerservaring is vaak frustrerend
\end{itemize}

Conversationele AI-systemen daarentegen begrijpen natuurlijke taal en kunnen contextbewust reageren. Large Language Models (LLMs) zoals GPT-4 hebben deze transitie mogelijk gemaakt door hun vermogen om menselijke taal te begrijpen en te genereren~\autocite{OpenAI2024}.

\subsection{Function calling: de brug tussen taal en actie}

Een cruciale ontwikkeling voor dit project is OpenAI's function calling mechanisme. Dit stelt het taalmodel in staat om niet alleen tekst te genereren, maar ook gestructureerde acties uit te voeren. Het model kan herkennen wanneer een gebruiker een specifieke actie wil (bijvoorbeeld een afspraak annuleren) en de juiste functie aanroepen met de correcte parameters~\autocite{OpenAI2024}.

\section{Twilio: programmeerbare telefonie}

Twilio is een cloud communications platform dat API's aanbiedt voor spraak, SMS, en andere communicatiekanalen. Voor dit project zijn de volgende Twilio-componenten relevant~\autocite{Twilio2024}:

\subsection{Programmable Voice}

Twilio's Programmable Voice API maakt het mogelijk om:

\begin{itemize}
    \item Inkomende oproepen te ontvangen op een virtueel telefoonnummer
    \item Oproepen te beantwoorden met vooraf opgenomen of dynamisch gegenereerde audio
    \item Spraakherkenning (Speech-to-Text) uit te voeren
    \item Text-to-Speech te gebruiken voor gesproken antwoorden
    \item Oproepen door te verbinden naar andere nummers
\end{itemize}

\subsection{TwiML: Twilio Markup Language}

De communicatie met Twilio verloopt via TwiML, een XML-gebaseerde markup language. Een typisch TwiML-antwoord ziet er als volgt uit:

\begin{listing}
\begin{minted}{xml}
<?xml version="1.0" encoding="UTF-8"?>
<Response>
    <Say voice="Polly.Ruben" language="nl-NL">
        Welkom bij de tandartspraktijk. Waarmee kan ik u helpen?
    </Say>
    <Gather input="speech"
           speechTimeout="auto"
           language="nl-NL"
           action="/api/twilio/process-speech">
    </Gather>
</Response>
\end{minted}
\caption[TwiML voorbeeld]{Voorbeeld van een TwiML-response met spraakherkenning}
\end{listing}

\subsection{Webhooks en real-time verwerking}

Twilio communiceert met de backend via webhooks. Bij elke fase van het gesprek (nieuwe oproep, spraakherkenning voltooid, oproep beëindigd) stuurt Twilio een HTTP POST request naar de geconfigureerde endpoint. Dit vereist dat de backend snel kan reageren—Twilio hanteert een timeout van 15 seconden~\autocite{Twilio2024}.

\section{OpenAI en Large Language Models}

OpenAI's GPT-modellen vormen het hart van de AI-assistent. Deze modellen zijn getraind op grote hoeveelheden tekstdata en kunnen menselijke taal begrijpen en genereren.

\subsection{Het GPT-model}

GPT (Generative Pre-trained Transformer) is een type neuraal netwerk dat geoptimaliseerd is voor taaltaken. Het model werkt door te voorspellen wat het volgende woord in een zin zou moeten zijn, gegeven de voorgaande context. Door miljarden parameters en enorme trainingsdata kan het model:

\begin{itemize}
    \item Natuurlijke conversaties voeren
    \item Context over meerdere berichten onthouden
    \item Intenties herkennen uit ongestructureerde input
    \item Meertalige ondersteuning bieden
\end{itemize}

\subsection{Function calling in detail}

Het function calling mechanisme werkt als volgt~\autocite{OpenAI2024}:

\begin{enumerate}
    \item De ontwikkelaar definieert beschikbare functies met hun parameters in JSON Schema formaat
    \item Het model ontvangt de gebruikersinput samen met de functiedefinities
    \item Het model bepaalt of een functie moet worden aangeroepen en met welke argumenten
    \item De backend voert de functie uit en geeft het resultaat terug aan het model
    \item Het model formuleert een natuurlijk antwoord voor de gebruiker
\end{enumerate}

Een voorbeeld van een functiedefinitie:

\begin{listing}
\begin{minted}{json}
{
    "name": "cancel_appointment",
    "description": "Annuleert de geselecteerde afspraak",
    "parameters": {
        "type": "object",
        "properties": {
            "reservationId": {
                "type": "string",
                "description": "Het unieke ID van de afspraak"
            }
        },
        "required": ["reservationId"]
    }
}
\end{minted}
\caption[Function calling definitie]{Voorbeeld van een function calling definitie voor afspraakannulering}
\end{listing}

\section{ASP.NET Core en Azure}

De backend van het systeem is gebouwd met ASP.NET Core, Microsoft's cross-platform framework voor webapplicaties. De keuze voor dit framework is gebaseerd op:

\begin{itemize}
    \item Integratie met de bestaande TurnUp-infrastructuur (ook .NET-gebaseerd)
    \item Uitstekende ondersteuning voor asynchrone verwerking
    \item Native Azure-integratie
    \item Hoge performance voor real-time toepassingen
\end{itemize}

\subsection{Azure-services}

Het systeem maakt gebruik van verschillende Azure-services~\autocite{Microsoft2024}:

\begin{itemize}
    \item \textbf{Azure App Service}: Hosting van de webapplicatie
    \item \textbf{Azure Key Vault}: Veilige opslag van API-sleutels en secrets
    \item \textbf{Azure Blob Storage}: Opslag van gespreksopnames
    \item \textbf{Azure Monitor}: Logging en monitoring van het systeem
\end{itemize}

\section{Vergelijkbare systemen}

Er bestaan reeds verschillende commerciële oplossingen voor AI-gestuurde telefonie:

\begin{itemize}
    \item \textbf{Google Dialogflow}: Biedt conversationele AI maar vereist aparte telefonie-integratie
    \item \textbf{Amazon Connect}: Volledige callcenter-oplossing maar complexer en duurder
    \item \textbf{Twilio Flex}: Contact center platform maar vereist meer configuratie
\end{itemize}

Het voordeel van een custom oplossing met Twilio en OpenAI is de volledige controle over de gebruikerservaring en de mogelijkheid tot diepe integratie met bestaande systemen zoals de TurnUp API.

\section{Uitdagingen en overwegingen}

Bij het bouwen van een AI-gestuurd callcenter moeten verschillende uitdagingen worden aangepakt:

\subsection{Latency}

De totale tijd tussen het moment dat een gebruiker spreekt en het moment dat het antwoord wordt afgespeeld bestaat uit:

\begin{enumerate}
    \item Spraakherkenning (Speech-to-Text): 0.5--2 seconden
    \item API-aanroep naar OpenAI: 1--3 seconden
    \item Text-to-Speech generatie: 0.3--1 seconde
\end{enumerate}

Om binnen Twilio's 15-seconden timeout te blijven, is optimalisatie cruciaal. Het gebruik van streaming responses en caching van veelvoorkomende antwoorden kan helpen.

\subsection{Foutafhandeling}

Een robuust systeem moet omgaan met:

\begin{itemize}
    \item Onverstaanbare spraak (achtergrondlawaai, accenten)
    \item Onverwachte gebruikersintentie
    \item API-fouten van Twilio of OpenAI
    \item Netwerkproblemen
\end{itemize}

\subsection{Privacy en veiligheid}

Telefoongesprekken bevatten persoonlijke gegevens en vallen onder de GDPR. Het systeem moet:

\begin{itemize}
    \item Gebruikers informeren over opname
    \item Data veilig opslaan en verwerken
    \item Toegang tot gegevens beperken tot geautoriseerde partijen
\end{itemize}

%%=============================================================================
%% Methodologie
%%=============================================================================

\chapter{\IfLanguageName{dutch}{Methodologie}{Methodology}}%
\label{ch:methodologie}

Dit hoofdstuk beschrijft de aanpak en methodologie die werd gevolgd tijdens de ontwikkeling van het AI-gestuurde callcenter. Het project werd opgedeeld in verschillende fasen, elk met specifieke doelstellingen en deliverables.

\section{Projectaanpak}

De ontwikkeling volgde een iteratieve aanpak, waarbij het systeem stapsgewijs werd opgebouwd en getest. Deze methode werd gekozen omdat:

\begin{itemize}
    \item Real-time spraakverwerking moeilijk vooraf volledig te specificeren is
    \item Gebruikersfeedback cruciaal is voor het verfijnen van de AI-prompts
    \item De integratie met externe systemen (Twilio, OpenAI) onvoorziene uitdagingen kan opleveren
\end{itemize}

\section{Fase 1: Literatuurstudie en technologiekeuze}

\subsection{Doelstelling}
Onderzoeken welke technologieën het meest geschikt zijn voor het bouwen van een AI-gestuurd callcenter.

\subsection{Aanpak}
\begin{itemize}
    \item Vergelijkende analyse van telefonie-platformen (Twilio, Vonage, Amazon Connect)
    \item Evaluatie van AI-modellen (OpenAI GPT, Google PaLM, Anthropic Claude)
    \item Onderzoek naar best practices voor conversationele AI
\end{itemize}

\subsection{Resultaat}
De keuze viel op Twilio voor telefonie vanwege de uitgebreide documentatie en flexibele API's, en OpenAI GPT vanwege de superieure taalcapaciteiten en het function calling mechanisme.

\section{Fase 2: Proof of Concept}

\subsection{Doelstelling}
Valideren dat de gekozen technologieën kunnen samenwerken voor een basis spraakgestuurde assistent.

\subsection{Aanpak}
\begin{enumerate}
    \item Opzetten van een Twilio-account met een Belgisch telefoonnummer
    \item Configureren van webhooks naar een lokale ontwikkelomgeving (via ngrok)
    \item Implementeren van een minimale ASP.NET Core API
    \item Integreren van OpenAI voor een simpele vraag-antwoord flow
\end{enumerate}

\subsection{Resultaat}
Een werkend prototype dat inkomende oproepen kon beantwoorden en eenvoudige vragen kon beantwoorden. Dit bewees de haalbaarheid van de gekozen architectuur.

\section{Fase 3: Systeemarchitectuur}

\subsection{Doelstelling}
Ontwerpen van een schaalbare en onderhoudbare architectuur voor het volledige systeem.

\subsection{Aanpak}
De architectuur werd ontworpen met de volgende componenten:

\begin{itemize}
    \item \textbf{TwilioController}: Ontvangt webhooks van Twilio en genereert TwiML responses
    \item \textbf{OpenAIChatService}: Beheert de communicatie met de OpenAI API
    \item \textbf{CallContext}: Houdt de staat van elk gesprek bij (conversatiegeschiedenis, acties)
    \item \textbf{Tool handlers}: Implementeren de business logic voor elke AI-functie
    \item \textbf{TurnUp API client}: Communiceert met de bestaande praktijksoftware
\end{itemize}

\subsection{Resultaat}
Een gelaagde architectuur die separation of concerns respecteert en eenvoudig uitbreidbaar is.

\section{Fase 4: Implementatie kernfunctionaliteiten}

\subsection{Doelstelling}
Implementeren van de vijf kernfuncties: klant opzoeken, afspraak annuleren, afspraak verzetten, wachtlijst, en doorverbinden.

\subsection{Aanpak}
Elke functie werd apart geïmplementeerd en getest:

\begin{enumerate}
    \item Definiëren van de function calling schema's voor OpenAI
    \item Implementeren van de tool handlers in C\#
    \item Koppelen aan de TurnUp API
    \item Testen met gesimuleerde gesprekken
    \item Verfijnen van de AI-prompts op basis van testresultaten
\end{enumerate}

\subsection{Resultaat}
Alle vijf functies werden succesvol geïmplementeerd en geïntegreerd in de conversatieflow.

\section{Fase 5: Prompt engineering}

\subsection{Doelstelling}
Optimaliseren van de system prompt zodat de AI-assistent zich gedraagt als een professionele en behulpzame medewerker.

\subsection{Aanpak}
De prompt werd iteratief verfijnd op basis van:

\begin{itemize}
    \item Testgesprekken met collega's en stakeholders
    \item Analyse van edge cases en foutieve interpretaties
    \item Feedback van TurnUp over gewenst gedrag
\end{itemize}

Belangrijke aspecten van de prompt:
\begin{itemize}
    \item Taalregels: strikt in de gekozen taal blijven
    \item Gedragsregels: korte zinnen, één vraag per keer
    \item Conversatieflows: gedetailleerde instructies per scenario
    \item Foutafhandeling: wat te doen bij onverwachte input
\end{itemize}

\subsection{Resultaat}
Een uitgebreide system prompt van meer dan 400 regels die alle scenario's afdekt.

\section{Fase 6: Meertalige ondersteuning}

\subsection{Doelstelling}
Implementeren van ondersteuning voor Nederlands, Engels en Frans.

\subsection{Aanpak}
\begin{enumerate}
    \item IVR-menu voor taalkeuze bij start van het gesprek
    \item Dynamische system prompt gebaseerd op gekozen taal
    \item Configuratie van Twilio's spraakherkenning per taal
    \item Selectie van geschikte Text-to-Speech stemmen (Amazon Polly)
\end{enumerate}

\subsection{Resultaat}
Het systeem ondersteunt drie talen met voor elke taal een passende stem en spraakherkenning.

\section{Fase 7: Testing en verfijning}

\subsection{Doelstelling}
Identificeren en oplossen van edge cases en bugs.

\subsection{Aanpak}
\begin{itemize}
    \item Handmatige tests met diverse scenario's
    \item Tests met echte telefoonnummers
    \item Analyse van gespreksopnames en transcripties
    \item Iteratieve verbetering van prompts en code
\end{itemize}

\subsection{Resultaat}
Een robuust systeem dat de meeste scenario's correct afhandelt.

\section{Gebruikte tools en technologieën}

\begin{table}[h]
\centering
\begin{tabular}{ll}
\toprule
\textbf{Categorie} & \textbf{Technologie} \\
\midrule
Programmeertaal & C\# (.NET 8) \\
Framework & ASP.NET Core \\
IDE & Visual Studio Code, Rider \\
Telefonie & Twilio Programmable Voice \\
AI & OpenAI GPT-4 \\
Hosting & Microsoft Azure \\
Versiebeheer & Git, GitHub \\
Lokale tunneling & ngrok \\
\bottomrule
\end{tabular}
\caption[Gebruikte technologieën]{Overzicht van gebruikte tools en technologieën}
\end{table}


% Voeg hier je eigen hoofdstukken toe die de ``corpus'' van je graduaatsproef
% vormen. De structuur en titels hangen af van je eigen onderzoek. Je kan bv.
% elke fase in je onderzoek in een apart hoofdstuk bespreken.

%%=============================================================================
%% Implementatie
%%=============================================================================

\chapter{Implementatie}%
\label{ch:implementatie}

Dit hoofdstuk beschrijft de technische implementatie van het AI-gestuurde callcenter in detail. We bespreken de systeemarchitectuur, de integratie met Twilio en OpenAI, en de werking van de verschillende componenten.

\section{Systeemarchitectuur}

Het systeem bestaat uit verschillende lagen die samenwerken om een naadloze gebruikerservaring te bieden. Figuur~\ref{fig:architectuur} toont een overzicht van de architectuur.

\begin{figure}[h]
\centering
\fbox{\parbox{0.8\textwidth}{
\centering
\textbf{Gespreksflow}\\[1em]
Patiënt belt $\rightarrow$ Twilio $\rightarrow$ ASP.NET Core API $\rightarrow$ OpenAI GPT\\[0.5em]
$\downarrow$\\[0.5em]
TurnUp API $\leftarrow$ Tool Handlers $\leftarrow$ Function Calling\\[0.5em]
$\downarrow$\\[0.5em]
TwiML Response $\rightarrow$ Twilio $\rightarrow$ Patiënt hoort antwoord
}}
\caption[Systeemarchitectuur]{Overzicht van de systeemarchitectuur en gespreksflow}
\label{fig:architectuur}
\end{figure}

\section{Twilio Controller}

De \texttt{TwilioController} is het ingangspunt voor alle Twilio webhooks. De belangrijkste endpoints zijn:

\begin{listing}
\begin{minted}{csharp}
[ApiController]
[Route("api/[controller]")]
public class TwilioController : ControllerBase
{
    [HttpPost("incoming-call")]
    public async Task<IActionResult> IncomingCall()
    {
        // Genereer IVR menu voor taalkeuze
        var response = new VoiceResponse();
        response.Say("Press 1 for English. Druk 2 voor Nederlands. " +
                     "Appuyez sur 3 pour le français.",
                     voice: "Polly.Joanna");
        response.Gather(
            numDigits: 1,
            action: new Uri("/api/twilio/language-selected", UriKind.Relative)
        );
        return Content(response.ToString(), "application/xml");
    }

    [HttpPost("process-speech/{conversationId}")]
    public async Task<IActionResult> ProcessSpeech(
        string conversationId,
        [FromForm] string SpeechResult)
    {
        // Verwerk spraak via OpenAI en genereer antwoord
        var aiResponse = await _openAiService
            .ProcessUserInput(conversationId, SpeechResult);
        return GenerateTwimlResponse(aiResponse);
    }
}
\end{minted}
\caption[TwilioController]{De TwilioController met incoming call en speech processing endpoints}
\end{listing}

\section{OpenAI Integratie}

De communicatie met OpenAI verloopt via de \texttt{OpenAIChatService}. Deze service beheert de conversatiegeschiedenis en het function calling mechanisme.

\subsection{System Prompt}

De system prompt is cruciaal voor het gedrag van de AI-assistent. Een vereenvoudigd voorbeeld:

\begin{listing}
\begin{minted}{csharp}
public static string GetSystemMessage(string locale)
{
    var languageInstruction = locale switch
    {
        "nl-be" => "You MUST speak ONLY in Dutch.",
        "fr-fr" => "You MUST speak ONLY in French.",
        _ => "You MUST speak ONLY in English."
    };

    return $@"{languageInstruction}

You are a friendly AI assistant for a dental practice call center.

PRIMARY GOAL: Help patients view/cancel appointments or request new ones.

TOOLS:
- lookup_customer(phoneNumber): Get patient info
- cancel_appointment(reservationId): Cancel appointment
- reschedule_appointment(date, time): Reschedule
- add_to_waitlist(reason, days): Add to waitlist
- transfer_to_practice(): Transfer call

BEHAVIOR: Short sentences. One question at a time. Never invent data.";
}
\end{minted}
\caption[System Prompt]{Vereenvoudigde versie van de system prompt}
\end{listing}

\subsection{Function Calling}

De AI-tools worden gedefinieerd als JSON Schema's die aan OpenAI worden meegegeven:

\begin{listing}
\begin{minted}{csharp}
private static readonly FunctionDefinition LookupCustomerFunction = new()
{
    Name = "lookup_customer",
    Description = "Look up customer information by phone number",
    Parameters = BinaryData.FromObjectAsJson(new
    {
        type = "object",
        properties = new
        {
            phoneNumber = new
            {
                type = "string",
                description = "The phone number to look up"
            }
        },
        required = new[] { "phoneNumber" }
    })
};
\end{minted}
\caption[Function Definition]{Definitie van de lookup\_customer functie voor OpenAI}
\end{listing}

\section{Tool Handlers}

Elke AI-tool heeft een bijbehorende handler die de business logic implementeert. Hieronder een voorbeeld van de \texttt{CancelAppointmentTool}:

\begin{listing}
\begin{minted}{csharp}
public class CancelAppointmentTool : ITool
{
    private readonly ITurnUpApiService _apiService;

    public async Task<string> ExecuteAsync(
        CallContext context,
        JsonElement arguments)
    {
        // Haal de geselecteerde afspraak op
        var appointment = context.GetSelectedAppointment();
        if (appointment == null)
            return "ERROR: No appointment selected";

        // Annuleer via TurnUp API
        var result = await _apiService.CancelReservationAsync(
            appointment.ReservationId,
            context.CustomerId
        );

        if (result.Success)
        {
            context.AddAction($"CANCELLED:{appointment.ReservationId}");

            // Check of reschedule mogelijk is
            if (result.RescheduleData != null)
            {
                context.StoreRescheduleData(result.RescheduleData);
                return "Appointment cancelled. Reschedule data available.";
            }
            return "Appointment cancelled successfully.";
        }

        return $"ERROR: {result.ErrorMessage}";
    }
}
\end{minted}
\caption[Cancel Appointment Tool]{Implementatie van de afspraak annulering tool}
\end{listing}

\section{Conversatie Context}

De \texttt{CallContext} klasse houdt alle relevante informatie bij gedurende een gesprek:

\begin{listing}
\begin{minted}{csharp}
public class CallContext
{
    public string ConversationId { get; set; }
    public string CallStatus { get; set; }
    public List<ChatMessage> Messages { get; set; } = new();
    public List<string> Actions { get; set; } = new();
    public string UserLocale { get; set; }

    // Opgeslagen data uit tool calls
    public CustomerInfo? Customer { get; private set; }
    public List<Appointment> Appointments { get; private set; }
    public Appointment? SelectedAppointment { get; private set; }
    public string? PracticePhone { get; private set; }

    public void AddAction(string action)
    {
        Actions.Add(action);
    }

    public string GetDynamicContext()
    {
        // Genereer context voor de AI gebaseerd op huidige staat
        var sb = new StringBuilder();
        sb.AppendLine("=== CURRENT STATE ===");

        if (Customer != null)
            sb.AppendLine($"Customer: {Customer.Name}");
        if (Appointments?.Any() == true)
            sb.AppendLine($"Appointments: {Appointments.Count}");

        return sb.ToString();
    }
}
\end{minted}
\caption[Call Context]{De CallContext klasse voor het bijhouden van gespreksstaat}
\end{listing}

\section{TwiML Response Generatie}

Na verwerking door OpenAI wordt een TwiML response gegenereerd die Twilio instrueert wat te doen:

\begin{listing}
\begin{minted}{csharp}
private IActionResult GenerateTwimlResponse(
    AiResponse response,
    CallContext context)
{
    var voiceResponse = new VoiceResponse();

    // Selecteer stem op basis van taal
    var voice = context.UserLocale switch
    {
        "nl-be" => "Polly.Ruben",
        "fr-fr" => "Polly.Mathieu",
        _ => "Polly.Joanna"
    };

    // Spreek het AI-antwoord uit
    voiceResponse.Say(response.Message, voice: voice);

    // Bepaal volgende actie
    if (response.ShouldEndCall)
    {
        voiceResponse.Hangup();
    }
    else if (response.ShouldTransfer)
    {
        voiceResponse.Dial(context.PracticePhone);
    }
    else
    {
        // Luister naar volgende input
        voiceResponse.Gather(
            input: new[] { Gather.InputEnum.Speech },
            speechTimeout: "auto",
            language: GetTwilioLanguage(context.UserLocale),
            action: new Uri($"/api/twilio/process-speech/{context.ConversationId}",
                           UriKind.Relative)
        );
    }

    return Content(voiceResponse.ToString(), "application/xml");
}
\end{minted}
\caption[TwiML Response]{Generatie van TwiML response op basis van AI-output}
\end{listing}

\section{Gespreksopname en Transcriptie}

Alle gesprekken worden opgenomen voor kwaliteitscontrole. De opname wordt gestart bij het begin van het gesprek:

\begin{listing}
\begin{minted}{csharp}
// In de incoming call handler
voiceResponse.Record(
    recordingStatusCallback: new Uri("/api/twilio/recording-callback",
                                     UriKind.Relative),
    recordingChannels: "dual",
    trim: "trim-silence"
);
\end{minted}
\caption[Recording]{Configuratie van gespreksopname in TwiML}
\end{listing}

Na afloop van het gesprek wordt de opname opgeslagen in Azure Blob Storage en wordt een transcriptie aangevraagd via de OpenAI Whisper API.

\section{Error Handling}

Robuuste foutafhandeling is essentieel voor een betrouwbaar systeem:

\begin{listing}
\begin{minted}{csharp}
public async Task<IActionResult> ProcessSpeech(...)
{
    try
    {
        // Timeout instellen (Twilio max = 15 sec)
        using var cts = new CancellationTokenSource(
            TimeSpan.FromSeconds(14));

        var response = await _openAiService
            .ProcessUserInputAsync(conversationId, speechResult, cts.Token);

        return GenerateTwimlResponse(response);
    }
    catch (OperationCanceledException)
    {
        // Timeout - vraag gebruiker om te herhalen
        return GenerateFallbackResponse(
            "Sorry, ik kon u niet verwerken. Kunt u dat herhalen?");
    }
    catch (Exception ex)
    {
        _logger.LogError(ex, "Error processing speech");
        return GenerateFallbackResponse(
            "Er is een fout opgetreden. Probeer het opnieuw.");
    }
}
\end{minted}
\caption[Error Handling]{Foutafhandeling met timeout en fallback responses}
\end{listing}

\section{Deployment}

Het systeem wordt gedeployed op Azure met de volgende configuratie:

\begin{itemize}
    \item \textbf{Azure App Service}: Hosting van de ASP.NET Core applicatie
    \item \textbf{Azure Key Vault}: Opslag van API keys (Twilio, OpenAI)
    \item \textbf{Azure Blob Storage}: Opslag van gespreksopnames
    \item \textbf{Azure Monitor}: Logging en alerting
\end{itemize}

De deployment wordt geautomatiseerd via GitHub Actions, waarbij elke push naar de main branch automatisch wordt gedeployed naar de productieomgeving.

%%=============================================================================
%% Resultaten
%%=============================================================================

\chapter{Resultaten}%
\label{ch:resultaten}

Dit hoofdstuk presenteert de resultaten van het ontwikkelde AI-callcenter en bespreekt de opgedane leerpunten tijdens het project.

\section{Functionele resultaten}

Het ontwikkelde systeem voldoet aan alle gestelde functionele eisen:

\begin{table}[h]
\centering
\begin{tabular}{lcc}
\toprule
\textbf{Functionaliteit} & \textbf{Geïmplementeerd} & \textbf{Getest} \\
\midrule
Klantgegevens opzoeken & Ja & Ja \\
Afspraak annuleren & Ja & Ja \\
Afspraak verzetten & Ja & Ja \\
Toevoegen aan wachtlijst & Ja & Ja \\
Doorverbinden naar praktijk & Ja & Ja \\
Meertalige ondersteuning (NL/EN/FR) & Ja & Ja \\
Gespreksopname & Ja & Ja \\
Transcriptie & Ja & Ja \\
\bottomrule
\end{tabular}
\caption[Functionele resultaten]{Overzicht van geïmplementeerde en geteste functionaliteiten}
\end{table}

\subsection{Prestaties}

De belangrijkste prestatie-indicatoren van het systeem:

\begin{itemize}
    \item \textbf{Beschikbaarheid}: 24/7 zonder menselijke tussenkomst
    \item \textbf{Gemiddelde gespreksduur}: minder dan 2 minuten voor standaard scenario's
    \item \textbf{Latency}: gemiddeld 2-4 seconden tussen vraag en antwoord
    \item \textbf{Succesratio}: circa 85\% van de gesprekken wordt succesvol afgehandeld
\end{itemize}

\section{Technische bevindingen}

\subsection{Uitdagingen en oplossingen}

Tijdens de ontwikkeling werden verschillende technische uitdagingen overwonnen:

\subsubsection{Twilio Timeout}

\textbf{Probleem}: Twilio hanteert een strikte timeout van 15 seconden voor webhook responses. Bij complexe OpenAI-aanroepen werd deze soms overschreden.

\textbf{Oplossing}: Implementatie van een interne timeout van 14 seconden met fallback naar een standaard ``Kunt u dat herhalen?'' response.

\subsubsection{Spraakherkenning van telefoonnummers}

\textbf{Probleem}: Telefoonnummers worden door spraakherkenning vaak incorrect getranscribeerd (bijvoorbeeld ``nul vier'' in plaats van ``04'').

\textbf{Oplossing}: Uitgebreide normalisatie-logica die verschillende formaten herkent en converteert naar het standaard E.164 formaat.

\subsubsection{Context behoud over meerdere beurten}

\textbf{Probleem}: De AI vergat soms eerder opgezochte informatie wanneer de gebruiker van onderwerp wisselde.

\textbf{Oplossing}: Implementatie van een dynamische context die bij elke beurt aan de prompt wordt toegevoegd.

\subsubsection{Meertalige prompts}

\textbf{Probleem}: De AI wisselde soms ongewild van taal of mengde talen.

\textbf{Oplossing}: Strikte taalinstructies aan het begin van de prompt en dynamische taalspecifieke voorbeeldzinnen.

\section{Leerpunten}

Dit project heeft geleid tot significante persoonlijke en professionele groei:

\subsection{Technische vaardigheden}

\begin{itemize}
    \item \textbf{Webhooks}: Diepgaande ervaring met asynchroon communiceren tussen externe services en de backend. Het begrijpen van callback-patronen en het correct afhandelen van meerdere gelijktijdige requests.
    \item \textbf{AI-integratie}: Praktische ervaring met prompt engineering en OpenAI's function calling mechanisme.
    \item \textbf{Real-time systemen}: Inzicht in de uitdagingen van lage-latency spraakverwerking en timeout-management.
    \item \textbf{Cloud deployment}: Hands-on ervaring met Azure-services, met name Key Vault voor secrets management.
\end{itemize}

\subsection{Professionele vaardigheden}

\begin{itemize}
    \item \textbf{Zelfstandig werken}: Het project vereiste veel eigen onderzoek en probleemoplossend denken.
    \item \textbf{Communicatie met stakeholders}: Regelmatig overleg met TurnUp over requirements.
    \item \textbf{Iteratief ontwikkelen}: Het belang van kleine, testbare incrementen.
\end{itemize}

\section{Vergelijking met doelstellingen}

\begin{table}[h]
\centering
\begin{tabular}{p{5cm}cc}
\toprule
\textbf{Doelstelling} & \textbf{Doel} & \textbf{Bereikt} \\
\midrule
Zonder menselijke tussenkomst & Ja & Ja \\
Gemiddelde gespreksduur & < 3 min & < 2 min \\
24/7 beschikbaarheid & Ja & Ja \\
Latency vraag-antwoord & < 2 sec & 2-4 sec* \\
Meertalige ondersteuning & 3 talen & 3 talen \\
\bottomrule
\end{tabular}
\caption[Doelstellingen]{Vergelijking doelstellingen. *Acceptabel voor gebruikers.}
\end{table}

\section{Beperkingen}

Het huidige systeem kent enkele beperkingen:

\begin{itemize}
    \item \textbf{Complexe vragen}: Vragen buiten het afspraakdomein kunnen niet worden beantwoord
    \item \textbf{Accenten}: Sterke accenten kunnen de spraakherkenning bemoeilijken
    \item \textbf{Achtergrondlawaai}: In lawaaiige omgevingen daalt de kwaliteit
\end{itemize}


%%=============================================================================
%% Conclusie
%%=============================================================================

\chapter{Conclusie}%
\label{ch:conclusie}

Dit afsluitende hoofdstuk beantwoordt de onderzoeksvragen, reflecteert op het behaalde resultaat, en geeft aanbevelingen voor toekomstig onderzoek.

\section{Beantwoording onderzoeksvragen}

\subsection{Hoofdvraag}

\textit{Hoe kan een AI-gestuurd callcenter de telefonische dienstverlening van tandartspraktijken automatiseren met behoud van klanttevredenheid?}

Dit onderzoek toont aan dat een AI-gestuurd callcenter de telefonische dienstverlening van tandartspraktijken effectief kan automatiseren. Door de combinatie van Twilio voor telefonie-infrastructuur en OpenAI GPT voor natuurlijke taalverwerking is het mogelijk om een systeem te bouwen dat:

\begin{itemize}
    \item 24/7 beschikbaar is zonder menselijke tussenkomst
    \item Patiënten in natuurlijke taal te woord staat
    \item Concrete acties kan uitvoeren zoals afspraken annuleren en verzetten
    \item Meertalige ondersteuning biedt (Nederlands, Engels, Frans)
    \item Naadloos integreert met bestaande praktijksoftware
\end{itemize}

De klanttevredenheid wordt behouden door korte, duidelijke communicatie, het correct uitvoeren van de gevraagde acties, en de mogelijkheid om door te verbinden naar een medewerker wanneer nodig.

\subsection{Deelvragen}

\textbf{1. Welke technologieën zijn het meest geschikt voor spraakgestuurde AI-assistenten?}

Twilio bleek de meest geschikte keuze voor telefonie vanwege de uitgebreide API's, goede documentatie, en flexibele webhook-architectuur. OpenAI GPT onderscheidt zich door superieure taalcapaciteiten en het function calling mechanisme dat gestructureerde acties mogelijk maakt.

\textbf{2. Hoe kan OpenAI's function calling worden ingezet voor afspraakbeheer?}

Door functies te definiëren met duidelijke namen, beschrijvingen en parameters kan het model automatisch bepalen wanneer welke functie moet worden aangeroepen. De backend voert vervolgens de eigenlijke actie uit en communiceert het resultaat terug naar het model voor een natuurlijke response.

\textbf{3. Welke architectuur is nodig voor real-time spraakverwerking?}

Een webhook-gebaseerde architectuur waarbij Twilio HTTP-callbacks stuurt naar de backend bleek effectief. Belangrijke aspecten zijn: strikte timeout-management (14 seconden intern), asynchrone verwerking, en stateless design met externe context-opslag.

\textbf{4. Hoe kan meertalige ondersteuning worden geïmplementeerd?}

Door een IVR-menu voor taalkeuze, dynamische system prompts per taal, taalspecifieke spraakherkenning-configuratie, en geschikte Text-to-Speech stemmen (Amazon Polly) per taal.

\textbf{5. Welke edge cases moeten worden afgehandeld?}

Belangrijke edge cases zijn: onverstaanbare spraak, telefoonnummers in diverse formaten, wisselen van intentie mid-gesprek, meerdere afspraken bij één patiënt, en API-fouten. Deze vereisen uitgebreide foutafhandeling en retry-logica.

\section{Reflectie}

\subsection{Sterke punten}

Het ontwikkelde systeem heeft verschillende sterke punten:

\begin{itemize}
    \item \textbf{Schaalbaarheid}: Het systeem kan zonder aanpassingen meerdere gelijktijdige gesprekken afhandelen.
    \item \textbf{Uitbreidbaarheid}: Nieuwe functies kunnen worden toegevoegd door extra tools te definiëren.
    \item \textbf{Onderhoudbaarheid}: De gelaagde architectuur maakt het eenvoudig om componenten apart te updaten.
    \item \textbf{Kostenefficiëntie}: Na de initiële ontwikkeling zijn de operationele kosten beperkt tot API-gebruik.
\end{itemize}

\subsection{Verbeterpunten}

Er zijn ook aspecten die beter kunnen:

\begin{itemize}
    \item \textbf{Latency}: De huidige 2-4 seconden kan worden verbeterd door overstap naar WebSockets.
    \item \textbf{Complexe scenario's}: Het systeem worstelt met gecombineerde of onduidelijke vragen.
    \item \textbf{Monitoring}: Meer uitgebreide monitoring en alerting zou helpen bij proactief onderhoud.
\end{itemize}

\section{Aanbevelingen voor toekomstig onderzoek}

Op basis van dit onderzoek worden de volgende richtingen voor toekomstig werk aanbevolen:

\subsection{Technische uitbreidingen}

\begin{itemize}
    \item \textbf{WebSockets}: Migratie van webhooks naar WebSockets voor lagere latency en real-time streaming van audio.
    \item \textbf{Outbound calling}: Implementatie van proactieve afspraakherinneringen via uitgaande oproepen.
    \item \textbf{RAG-integratie}: Retrieval-Augmented Generation voor het beantwoorden van praktijkspecifieke vragen op basis van documentatie.
    \item \textbf{Voice cloning}: Custom stemmen die passen bij de identiteit van de praktijk.
\end{itemize}

\subsection{Functionele uitbreidingen}

\begin{itemize}
    \item \textbf{Meer zorgsectoren}: Uitbreiding naar huisartsen, kinesisten, en andere zorgverleners.
    \item \textbf{Intake-gesprekken}: Automatiseren van nieuwe patiënt intakes.
    \item \textbf{Noodgevallen-detectie}: Herkenning van urgente situaties en directe doorschakeling.
\end{itemize}

\subsection{Onderzoeksrichtingen}

\begin{itemize}
    \item \textbf{Gebruikerstevredenheid}: Kwantitatief onderzoek naar de tevredenheid van patiënten met AI-telefonie.
    \item \textbf{Kostenbatenanalyse}: Vergelijking van operationele kosten versus traditionele receptie.
    \item \textbf{Privacy-aspecten}: Diepgaand onderzoek naar GDPR-compliance en patiëntperceptie van data-gebruik.
\end{itemize}

\section{Slotwoord}

Dit project demonstreert dat de combinatie van moderne AI-technologie en cloud-telefonie een krachtig instrument is voor het automatiseren van klantenservice in de zorgsector. De technologie is rijp voor productie-implementatie, met de kanttekening dat menselijke backup voor complexe situaties essentieel blijft.

De opgedane kennis en ervaring op het gebied van webhooks, AI-integratie, real-time systemen, en cloud deployment vormen een waardevolle basis voor verdere professionele ontwikkeling. Dit project bewijst dat met de juiste technologiekeuzes en een iteratieve aanpak, zelfs complexe AI-toepassingen binnen het bereik liggen van een graduaatsstudent.

De toekomst van klantenservice in de zorg ligt in de combinatie van AI-efficiëntie en menselijke empathie. Dit project levert daar een concrete bijdrage aan.


%---------- Bijlagen -----------------------------------------------------------

%% Het onderzoeksvoorstel is niet opgenomen in dit document om binnen de
%% limiet van 10-15 pagina's te blijven. Het voorstel is apart beschikbaar.

%\appendix

%\chapter{Onderzoeksvoorstel}

%Het onderwerp van deze graduaatsproef is gebaseerd op een onderzoeksvoorstel dat vooraf werd beoordeeld door de promotor. Dat voorstel is opgenomen in deze bijlage.

%% TODO:
%\section*{Samenvatting}

% Kopieer en plak hier de samenvatting (abstract) van je onderzoeksvoorstel.

% Verwijzing naar het bestand met de inhoud van het onderzoeksvoorstel
%%---------- Inleiding ---------------------------------------------------------

\section{Inleiding}%
\label{sec:inleiding}

De zorgsector ondergaat een snelle digitalisering, waarbij kunstmatige intelligentie (AI) een steeds belangrijkere rol speelt in het verbeteren van patiëntervaring en operationele efficiëntie. Tandartspraktijken vormen hierop geen uitzondering: zij ontvangen dagelijks tientallen telefoontjes van patiënten die informatie willen over hun afspraken, een afspraak willen annuleren of verzetten, of doorverbonden willen worden met de praktijk.

\subsection{Probleemstelling}

Het telefonisch afhandelen van afspraakgerelateerde vragen brengt verschillende uitdagingen met zich mee:

\begin{itemize}
    \item \textbf{Beperkte bereikbaarheid}: Patiënten kunnen enkel tijdens kantooruren terecht, terwijl veel mensen net dan aan het werk zijn.
    \item \textbf{Wachttijden}: Tijdens piekmomenten ontstaan er wachtrijen, wat leidt tot frustratie.
    \item \textbf{Repetitieve taken}: Receptionisten besteden veel tijd aan standaardvragen die potentieel geautomatiseerd kunnen worden.
    \item \textbf{Meertaligheid}: In België is ondersteuning voor Nederlands, Frans en Engels vaak noodzakelijk.
    \item \textbf{Kosten}: Extra personeel aannemen voor telefonische ondersteuning is kostbaar.
\end{itemize}

\subsection{Doelgroep}

Dit onderzoek richt zich op \textbf{TurnUp}, een Belgisch softwarebedrijf gespecialiseerd in praktijkbeheersoftware voor tandartsen. TurnUp zoekt naar innovatieve manieren om hun klanten te ondersteunen bij het verbeteren van de telefonische bereikbaarheid. Het ontwikkelde systeem zal worden geïntegreerd met de bestaande TurnUp API.

\subsection{Onderzoeksvraag}

De centrale onderzoeksvraag luidt:

\begin{quote}
\textit{Hoe kan een AI-gestuurd callcenter de telefonische dienstverlening van tandartspraktijken automatiseren met behoud van klanttevredenheid?}
\end{quote}

Hieruit vloeien de volgende deelvragen voort:
\begin{enumerate}
    \item Welke technologieën zijn het meest geschikt voor spraakgestuurde AI-assistenten?
    \item Hoe kan OpenAI's function calling mechanisme worden ingezet voor afspraakbeheer?
    \item Welke architectuur is nodig voor real-time spraakverwerking met acceptabele latency?
    \item Hoe kan meertalige ondersteuning worden geïmplementeerd?
\end{enumerate}

\subsection{Onderzoeksdoelstelling}

Het concrete eindresultaat van dit onderzoek is een \textbf{werkend proof-of-concept} van een AI-gestuurd inbound callcenter dat:
\begin{itemize}
    \item Inkomende oproepen automatisch beantwoordt via Twilio
    \item Patiënten in natuurlijke taal te woord staat via OpenAI GPT
    \item Afspraken kan opzoeken, annuleren en verzetten
    \item Drie talen ondersteunt (Nederlands, Engels, Frans)
    \item Integreert met de TurnUp API
\end{itemize}

%---------- Stand van zaken ---------------------------------------------------

\section{Literatuurstudie}%
\label{sec:literatuurstudie}

\subsection{Conversationele AI}

De evolutie van Interactive Voice Response (IVR) systemen naar conversationele AI markeert een significante verschuiving in klantenservice. Traditionele IVR-systemen werken met rigide keuzemenu's, terwijl moderne AI-systemen natuurlijke taal kunnen begrijpen en contextbewust kunnen reageren~\autocite{OpenAI2023}.

Large Language Models (LLMs) zoals GPT-4 hebben deze transitie mogelijk gemaakt. Deze modellen zijn getraind op enorme hoeveelheden tekstdata en kunnen menselijke taal begrijpen, genereren en zelfs gestructureerde acties uitvoeren via het function calling mechanisme~\autocite{OpenAI2024}.

\subsection{Twilio en programmeerbare telefonie}

Twilio is een cloud communications platform dat API's aanbiedt voor spraak, SMS en andere communicatiekanalen. De Programmable Voice API maakt het mogelijk om inkomende oproepen te ontvangen, spraakherkenning uit te voeren, en tekst om te zetten naar spraak~\autocite{Twilio2024}.

Twilio communiceert met backend-systemen via webhooks: bij elke fase van een gesprek stuurt Twilio een HTTP-request naar een geconfigureerde endpoint. Dit vereist dat de backend binnen 15 seconden kan reageren.

\subsection{OpenAI Function Calling}

Een recente ontwikkeling die cruciaal is voor dit project is OpenAI's function calling mechanisme. Dit stelt het taalmodel in staat om niet alleen tekst te genereren, maar ook gestructureerde functies aan te roepen met de juiste parameters~\autocite{OpenAI2024}. Hierdoor kan het model bijvoorbeeld herkennen dat een gebruiker een afspraak wil annuleren en automatisch de juiste API-aanroep triggeren.

\subsection{Vergelijkbare oplossingen}

Er bestaan reeds commerciële oplossingen voor AI-telefonie, zoals Google Dialogflow, Amazon Connect en Twilio Flex. Het voordeel van een custom oplossing is de volledige controle over de gebruikerservaring en de mogelijkheid tot diepe integratie met bestaande systemen zoals de TurnUp API.

%---------- Methodologie ------------------------------------------------------

\section{Methodologie}%
\label{sec:methodologie}

Het onderzoek wordt opgedeeld in de volgende fasen:

\subsection{Fase 1: Literatuurstudie (2 weken)}

\textbf{Doel}: Diepgaand onderzoek naar de state-of-the-art in conversationele AI en telefonie-integratie.

\textbf{Aanpak}: Analyse van documentatie van Twilio, OpenAI en vergelijkbare platformen. Evaluatie van best practices voor prompt engineering en function calling.

\textbf{Deliverable}: Rapport met technologiekeuze en architectuurvoorstel.

\subsection{Fase 2: Proof of Concept (3 weken)}

\textbf{Doel}: Valideren dat Twilio en OpenAI kunnen samenwerken voor een basis spraakassistent.

\textbf{Aanpak}:
\begin{itemize}
    \item Opzetten Twilio-account met Belgisch telefoonnummer
    \item Implementeren ASP.NET Core backend met webhooks
    \item Integreren OpenAI voor eenvoudige vraag-antwoord flow
\end{itemize}

\textbf{Deliverable}: Werkend prototype dat oproepen kan beantwoorden.

\subsection{Fase 3: Implementatie kernfunctionaliteiten (4 weken)}

\textbf{Doel}: Ontwikkelen van de vijf kernfuncties via function calling.

\textbf{Aanpak}:
\begin{itemize}
    \item Definiëren function calling schema's
    \item Implementeren tool handlers in C\#
    \item Koppelen aan TurnUp API
    \item Iteratief testen en verfijnen van prompts
\end{itemize}

\textbf{Deliverable}: Volledig functioneel systeem met alle kernfuncties.

\subsection{Fase 4: Meertalige ondersteuning (2 weken)}

\textbf{Doel}: Implementeren van Nederlands, Engels en Frans.

\textbf{Aanpak}: IVR-menu voor taalkeuze, taalspecifieke prompts, configuratie spraakherkenning per taal.

\textbf{Deliverable}: Meertalig systeem met passende Text-to-Speech stemmen.

\subsection{Fase 5: Testing en documentatie (3 weken)}

\textbf{Doel}: Uitgebreid testen en documenteren van het systeem.

\textbf{Aanpak}: Handmatige tests, analyse van edge cases, schrijven van technische documentatie en graduaatsproefverslag.

\textbf{Deliverable}: Getest systeem en afgewerkt verslag.

\subsection{Gebruikte tools}

\begin{itemize}
    \item \textbf{Programmeertaal}: C\# (.NET 8)
    \item \textbf{Framework}: ASP.NET Core
    \item \textbf{Telefonie}: Twilio Programmable Voice
    \item \textbf{AI}: OpenAI GPT-4 met function calling
    \item \textbf{Hosting}: Microsoft Azure
    \item \textbf{IDE}: Visual Studio Code, JetBrains Rider
\end{itemize}

%---------- Verwachte resultaten ----------------------------------------------

\section{Verwacht resultaat, conclusie}%
\label{sec:verwachte_resultaten}

\subsection{Verwachte resultaten}

Op basis van de literatuurstudie en vooronderzoek worden de volgende resultaten verwacht:

\begin{itemize}
    \item Een werkend AI-callcenter dat 24/7 beschikbaar is
    \item Succesvolle afhandeling van minimaal 80\% van de standaard scenario's (afspraak opzoeken, annuleren, verzetten)
    \item Gemiddelde gespreksduur van minder dan 3 minuten
    \item Latency van minder dan 5 seconden tussen vraag en antwoord
    \item Volledige meertalige ondersteuning (NL/EN/FR)
\end{itemize}

\subsection{Meerwaarde voor de doelgroep}

Dit onderzoek levert concrete meerwaarde voor TurnUp en hun klanten:

\begin{itemize}
    \item \textbf{Verbeterde bereikbaarheid}: Patiënten kunnen 24/7 terecht, ook buiten kantooruren.
    \item \textbf{Lagere werkdruk}: Receptionisten worden ontlast van repetitieve taken.
    \item \textbf{Kostenefficiëntie}: Geen extra personeel nodig voor telefonische ondersteuning.
    \item \textbf{Schaalbaarheid}: Het systeem kan meerdere gesprekken gelijktijdig afhandelen.
    \item \textbf{Innovatie}: TurnUp kan zich onderscheiden met een moderne AI-oplossing.
\end{itemize}

\subsection{Mogelijke uitbreidingen}

Als het basisproject succesvol is, zijn de volgende uitbreidingen mogelijk:
\begin{itemize}
    \item Outbound calling voor proactieve afspraakherinneringen
    \item Integratie met andere zorgsectoren (huisartsen, kinesisten)
    \item RAG-integratie voor beantwoorden van praktijkspecifieke vragen
\end{itemize}

\subsection{Conclusie}

De combinatie van Twilio voor telefonie en OpenAI voor natuurlijke taalverwerking biedt een veelbelovende basis voor het automatiseren van klantenservice in de zorgsector. Dit onderzoek zal aantonen of en hoe deze technologieën effectief kunnen worden ingezet voor tandartspraktijken, met concrete aanbevelingen voor verdere implementatie.


%%---------- Andere bijlagen --------------------------------------------------
% TODO: Voeg hier eventuele andere bijlagen toe. Bv. als je deze BP voor de
% tweede keer indient, een overzicht van de verbeteringen t.o.v. het origineel.
%\input{...}

%%---------- Backmatter, referentielijst ---------------------------------------

\backmatter{}

\setlength\bibitemsep{2pt} %% Add Some space between the bibliograpy entries
\nocite{*} %% Include all bibliography entries, not just cited ones
\printbibliography[heading=bibintoc]

\end{document}
