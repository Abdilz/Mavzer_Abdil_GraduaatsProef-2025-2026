%%=============================================================================
%% Methodologie
%%=============================================================================

\chapter{\IfLanguageName{dutch}{Methodologie}{Methodology}}%
\label{ch:methodologie}

Dit hoofdstuk beschrijft de aanpak en methodologie die werd gevolgd tijdens de ontwikkeling van het AI-gestuurde callcenter. Het project werd opgedeeld in verschillende fasen, elk met specifieke doelstellingen en deliverables.

\section{Projectaanpak}

De ontwikkeling volgde een iteratieve aanpak, waarbij het systeem stapsgewijs werd opgebouwd en getest. Deze methode werd gekozen omdat:

\begin{itemize}
    \item Real-time spraakverwerking moeilijk vooraf volledig te specificeren is
    \item Gebruikersfeedback cruciaal is voor het verfijnen van de AI-prompts
    \item De integratie met externe systemen (Twilio, OpenAI) onvoorziene uitdagingen kan opleveren
\end{itemize}

\section{Fase 1: Literatuurstudie en technologiekeuze}

\subsection{Doelstelling}
Onderzoeken welke technologieën het meest geschikt zijn voor het bouwen van een AI-gestuurd callcenter.

\subsection{Aanpak}
\begin{itemize}
    \item Vergelijkende analyse van telefonie-platformen (Twilio, Vonage, Amazon Connect)
    \item Evaluatie van AI-modellen (OpenAI GPT, Google PaLM, Anthropic Claude)
    \item Onderzoek naar best practices voor conversationele AI
\end{itemize}

\subsection{Resultaat}
De keuze viel op Twilio voor telefonie vanwege de uitgebreide documentatie en flexibele API's, en OpenAI GPT vanwege de superieure taalcapaciteiten en het function calling mechanisme.

\section{Fase 2: Proof of Concept}

\subsection{Doelstelling}
Valideren dat de gekozen technologieën kunnen samenwerken voor een basis spraakgestuurde assistent.

\subsection{Aanpak}
\begin{enumerate}
    \item Opzetten van een Twilio-account met een Belgisch telefoonnummer
    \item Configureren van webhooks naar een lokale ontwikkelomgeving (via ngrok)
    \item Implementeren van een minimale ASP.NET Core API
    \item Integreren van OpenAI voor een simpele vraag-antwoord flow
\end{enumerate}

\subsection{Resultaat}
Een werkend prototype dat inkomende oproepen kon beantwoorden en eenvoudige vragen kon beantwoorden. Dit bewees de haalbaarheid van de gekozen architectuur.

\section{Fase 3: Systeemarchitectuur}

\subsection{Doelstelling}
Ontwerpen van een schaalbare en onderhoudbare architectuur voor het volledige systeem.

\subsection{Aanpak}
De architectuur werd ontworpen met de volgende componenten:

\begin{itemize}
    \item \textbf{TwilioController}: Ontvangt webhooks van Twilio en genereert TwiML responses
    \item \textbf{OpenAIChatService}: Beheert de communicatie met de OpenAI API
    \item \textbf{CallContext}: Houdt de staat van elk gesprek bij (conversatiegeschiedenis, acties)
    \item \textbf{Tool handlers}: Implementeren de business logic voor elke AI-functie
    \item \textbf{TurnUp API client}: Communiceert met de bestaande praktijksoftware
\end{itemize}

\subsection{Resultaat}
Een gelaagde architectuur die separation of concerns respecteert en eenvoudig uitbreidbaar is.

\section{Fase 4: Implementatie kernfunctionaliteiten}

\subsection{Doelstelling}
Implementeren van de vijf kernfuncties: klant opzoeken, afspraak annuleren, afspraak verzetten, wachtlijst, en doorverbinden.

\subsection{Aanpak}
Elke functie werd apart geïmplementeerd en getest:

\begin{enumerate}
    \item Definiëren van de function calling schema's voor OpenAI
    \item Implementeren van de tool handlers in C\#
    \item Koppelen aan de TurnUp API
    \item Testen met gesimuleerde gesprekken
    \item Verfijnen van de AI-prompts op basis van testresultaten
\end{enumerate}

\subsection{Resultaat}
Alle vijf functies werden succesvol geïmplementeerd en geïntegreerd in de conversatieflow.

\section{Fase 5: Prompt engineering}

\subsection{Doelstelling}
Optimaliseren van de system prompt zodat de AI-assistent zich gedraagt als een professionele en behulpzame medewerker.

\subsection{Aanpak}
De prompt werd iteratief verfijnd op basis van:

\begin{itemize}
    \item Testgesprekken met collega's en stakeholders
    \item Analyse van edge cases en foutieve interpretaties
    \item Feedback van TurnUp over gewenst gedrag
\end{itemize}

Belangrijke aspecten van de prompt:
\begin{itemize}
    \item Taalregels: strikt in de gekozen taal blijven
    \item Gedragsregels: korte zinnen, één vraag per keer
    \item Conversatieflows: gedetailleerde instructies per scenario
    \item Foutafhandeling: wat te doen bij onverwachte input
\end{itemize}

\subsection{Resultaat}
Een uitgebreide system prompt van meer dan 400 regels die alle scenario's afdekt.

\section{Fase 6: Meertalige ondersteuning}

\subsection{Doelstelling}
Implementeren van ondersteuning voor Nederlands, Engels en Frans.

\subsection{Aanpak}
\begin{enumerate}
    \item IVR-menu voor taalkeuze bij start van het gesprek
    \item Dynamische system prompt gebaseerd op gekozen taal
    \item Configuratie van Twilio's spraakherkenning per taal
    \item Selectie van geschikte Text-to-Speech stemmen (Amazon Polly)
\end{enumerate}

\subsection{Resultaat}
Het systeem ondersteunt drie talen met voor elke taal een passende stem en spraakherkenning.

\section{Fase 7: Testing en verfijning}

\subsection{Doelstelling}
Identificeren en oplossen van edge cases en bugs.

\subsection{Aanpak}
\begin{itemize}
    \item Handmatige tests met diverse scenario's
    \item Tests met echte telefoonnummers
    \item Analyse van gespreksopnames en transcripties
    \item Iteratieve verbetering van prompts en code
\end{itemize}

\subsection{Resultaat}
Een robuust systeem dat de meeste scenario's correct afhandelt.

\section{Gebruikte tools en technologieën}

\begin{table}[h]
\centering
\begin{tabular}{ll}
\toprule
\textbf{Categorie} & \textbf{Technologie} \\
\midrule
Programmeertaal & C\# (.NET 8) \\
Framework & ASP.NET Core \\
IDE & Visual Studio Code, Rider \\
Telefonie & Twilio Programmable Voice \\
AI & OpenAI GPT-4 \\
Hosting & Microsoft Azure \\
Versiebeheer & Git, GitHub \\
Lokale tunneling & ngrok \\
\bottomrule
\end{tabular}
\caption[Gebruikte technologieën]{Overzicht van gebruikte tools en technologieën}
\end{table}
