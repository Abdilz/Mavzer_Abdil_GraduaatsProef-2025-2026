%%=============================================================================
%% Conclusie
%%=============================================================================

\chapter{Conclusie}%
\label{ch:conclusie}

\section{Beantwoording onderzoeksvragen}

\textit{Hoe kan een AI-gestuurd callcenter de telefonische dienstverlening van tandartspraktijken automatiseren met behoud van klanttevredenheid?}

Dit onderzoek toont aan dat de combinatie van Twilio en OpenAI GPT een effectief AI-callcenter mogelijk maakt dat:
\begin{itemize}
    \item 24/7 beschikbaar is zonder menselijke tussenkomst
    \item Patiënten in natuurlijke taal te woord staat (NL/EN/FR)
    \item Afspraken kan opzoeken, annuleren en verzetten
    \item Naadloos integreert met bestaande praktijksoftware
\end{itemize}

\textbf{Deelvragen:}
\begin{enumerate}
    \item \textbf{Technologieën}: Twilio voor telefonie (webhooks, STT/TTS), OpenAI GPT voor function calling
    \item \textbf{Function calling}: Functies definiëren met JSON Schema; model bepaalt automatisch welke aan te roepen
    \item \textbf{Architectuur}: Webhook-gebaseerd met 14 sec timeout, stateless design
    \item \textbf{Meertaligheid}: IVR-menu, dynamische prompts, taalspecifieke TTS-stemmen
\end{enumerate}

\section{Reflectie}

\textbf{Sterke punten:}
\begin{itemize}
    \item Schaalbaarheid en uitbreidbaarheid
    \item Gelaagde, onderhoudbare architectuur
    \item Kostenefficiënt na initiële ontwikkeling
\end{itemize}

\textbf{Verbeterpunten:}
\begin{itemize}
    \item Latency verlagen via WebSockets
    \item Complexere scenario's beter afhandelen
\end{itemize}

\section{Toekomstig werk}

\begin{itemize}
    \item \textbf{WebSockets}: Lagere latency, real-time audio streaming
    \item \textbf{Outbound calling}: Proactieve afspraakherinneringen
    \item \textbf{RAG-integratie}: Praktijkspecifieke vragen beantwoorden
    \item \textbf{Uitbreiding}: Andere zorgsectoren (huisartsen, kinesisten)
\end{itemize}

\section{Slotwoord}

Dit project demonstreert dat moderne AI-technologie en cloud-telefonie klantenservice in de zorgsector effectief kunnen automatiseren. De opgedane kennis op het gebied van webhooks, AI-integratie en cloud deployment vormt een waardevolle basis voor verdere professionele ontwikkeling.
