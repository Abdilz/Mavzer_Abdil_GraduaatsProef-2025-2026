%==============================================================================
% Sjabloon onderzoeksvoorstel bachproef
%==============================================================================
% Gebaseerd op document class `hogent-article'
% zie <https://github.com/HoGentTIN/latex-hogent-article>

\documentclass{hogent-article}

% Invoegen bibliografiebestand
\addbibresource{voorstel.bib}

% Informatie over de opleiding, het vak en soort opdracht
\studyprogramme{Graduaat in het Programmeren}
\course{Graduaatsproef}
\assignmenttype{Onderzoeksvoorstel}

\academicyear{2025-2026}

% Werktitel
\title{AI-gestuurd Callcenter voor Tandartspraktijken: Automatisering van patiëntcommunicatie met Twilio en OpenAI}

% Studentnaam en emailadres
\author{Abdil Mavzer}
\email{abdil.mavzer@student.hogent.be}

% Co-promotor
\supervisor[Co-promotor]{J. Decubber (TurnUp, \href{mailto:jona.decubber@turnup.be}{jona.decubber@turnup.be})}

% Specialisatierichting
\specialisation{Mobile \& Enterprise development}
\keywords{OpenAI, Twilio, Voice AI, ASP.NET Core, Function Calling}

\begin{document}

\begin{abstract}
Dit onderzoeksvoorstel beschrijft de ontwikkeling van een AI-gestuurd inbound callcenter voor tandartspraktijken. Tandartspraktijken besteden dagelijks aanzienlijke tijd aan telefonisch afspraakbeheer, wat leidt tot hoge werkdruk voor receptionisten en beperkte bereikbaarheid voor patiënten. De centrale onderzoeksvraag luidt: hoe kan een AI-gestuurd callcenter de telefonische dienstverlening automatiseren met behoud van klanttevredenheid? Het onderzoek richt zich op TurnUp, een Belgisch softwarebedrijf voor tandartspraktijken, als concrete casus. De methodologie omvat een literatuurstudie naar conversationele AI en telefonie-platformen, gevolgd door de ontwikkeling van een proof-of-concept met Twilio (telefonie) en OpenAI GPT (taalverwerking). Het verwachte resultaat is een werkend prototype dat 24/7 beschikbaar is, drie talen ondersteunt (NL/EN/FR), en afspraken kan beheren via OpenAI's function calling mechanisme. Dit onderzoek levert meerwaarde voor tandartspraktijken door de telefonische bereikbaarheid te verbeteren en administratieve lasten te verlagen.
\end{abstract}

\tableofcontents

% De hoofdtekst van het voorstel zit in een apart bestand, zodat het makkelijk
% kan opgenomen worden in de bijlagen van de graduaatsproef zelf.
%---------- Inleiding ---------------------------------------------------------

\section{Inleiding}%
\label{sec:inleiding}

De zorgsector ondergaat een snelle digitalisering, waarbij kunstmatige intelligentie (AI) een steeds belangrijkere rol speelt in het verbeteren van patiëntervaring en operationele efficiëntie. Tandartspraktijken vormen hierop geen uitzondering: zij ontvangen dagelijks tientallen telefoontjes van patiënten die informatie willen over hun afspraken, een afspraak willen annuleren of verzetten, of doorverbonden willen worden met de praktijk.

\subsection{Probleemstelling}

Het telefonisch afhandelen van afspraakgerelateerde vragen brengt verschillende uitdagingen met zich mee:

\begin{itemize}
    \item \textbf{Beperkte bereikbaarheid}: Patiënten kunnen enkel tijdens kantooruren terecht, terwijl veel mensen net dan aan het werk zijn.
    \item \textbf{Wachttijden}: Tijdens piekmomenten ontstaan er wachtrijen, wat leidt tot frustratie.
    \item \textbf{Repetitieve taken}: Receptionisten besteden veel tijd aan standaardvragen die potentieel geautomatiseerd kunnen worden.
    \item \textbf{Meertaligheid}: In België is ondersteuning voor Nederlands, Frans en Engels vaak noodzakelijk.
    \item \textbf{Kosten}: Extra personeel aannemen voor telefonische ondersteuning is kostbaar.
\end{itemize}

\subsection{Doelgroep}

Dit onderzoek richt zich op \textbf{TurnUp}, een Belgisch softwarebedrijf gespecialiseerd in praktijkbeheersoftware voor tandartsen. TurnUp zoekt naar innovatieve manieren om hun klanten te ondersteunen bij het verbeteren van de telefonische bereikbaarheid. Het ontwikkelde systeem zal worden geïntegreerd met de bestaande TurnUp API.

\subsection{Onderzoeksvraag}

De centrale onderzoeksvraag luidt:

\begin{quote}
\textit{Hoe kan een AI-gestuurd callcenter de telefonische dienstverlening van tandartspraktijken automatiseren met behoud van klanttevredenheid?}
\end{quote}

Hieruit vloeien de volgende deelvragen voort:
\begin{enumerate}
    \item Welke technologieën zijn het meest geschikt voor spraakgestuurde AI-assistenten?
    \item Hoe kan OpenAI's function calling mechanisme worden ingezet voor afspraakbeheer?
    \item Welke architectuur is nodig voor real-time spraakverwerking met acceptabele latency?
    \item Hoe kan meertalige ondersteuning worden geïmplementeerd?
\end{enumerate}

\subsection{Onderzoeksdoelstelling}

Het concrete eindresultaat van dit onderzoek is een \textbf{werkend proof-of-concept} van een AI-gestuurd inbound callcenter dat:
\begin{itemize}
    \item Inkomende oproepen automatisch beantwoordt via Twilio
    \item Patiënten in natuurlijke taal te woord staat via OpenAI GPT
    \item Afspraken kan opzoeken, annuleren en verzetten
    \item Drie talen ondersteunt (Nederlands, Engels, Frans)
    \item Integreert met de TurnUp API
\end{itemize}

%---------- Stand van zaken ---------------------------------------------------

\section{Literatuurstudie}%
\label{sec:literatuurstudie}

\subsection{Conversationele AI}

De evolutie van Interactive Voice Response (IVR) systemen naar conversationele AI markeert een significante verschuiving in klantenservice. Traditionele IVR-systemen werken met rigide keuzemenu's, terwijl moderne AI-systemen natuurlijke taal kunnen begrijpen en contextbewust kunnen reageren~\autocite{OpenAI2023}.

Large Language Models (LLMs) zoals GPT-4 hebben deze transitie mogelijk gemaakt. Deze modellen zijn getraind op enorme hoeveelheden tekstdata en kunnen menselijke taal begrijpen, genereren en zelfs gestructureerde acties uitvoeren via het function calling mechanisme~\autocite{OpenAI2024}.

\subsection{Twilio en programmeerbare telefonie}

Twilio is een cloud communications platform dat API's aanbiedt voor spraak, SMS en andere communicatiekanalen. De Programmable Voice API maakt het mogelijk om inkomende oproepen te ontvangen, spraakherkenning uit te voeren, en tekst om te zetten naar spraak~\autocite{Twilio2024}.

Twilio communiceert met backend-systemen via webhooks: bij elke fase van een gesprek stuurt Twilio een HTTP-request naar een geconfigureerde endpoint. Dit vereist dat de backend binnen 15 seconden kan reageren.

\subsection{OpenAI Function Calling}

Een recente ontwikkeling die cruciaal is voor dit project is OpenAI's function calling mechanisme. Dit stelt het taalmodel in staat om niet alleen tekst te genereren, maar ook gestructureerde functies aan te roepen met de juiste parameters~\autocite{OpenAI2024}. Hierdoor kan het model bijvoorbeeld herkennen dat een gebruiker een afspraak wil annuleren en automatisch de juiste API-aanroep triggeren.

\subsection{Vergelijkbare oplossingen}

Er bestaan reeds commerciële oplossingen voor AI-telefonie, zoals Google Dialogflow, Amazon Connect en Twilio Flex. Het voordeel van een custom oplossing is de volledige controle over de gebruikerservaring en de mogelijkheid tot diepe integratie met bestaande systemen zoals de TurnUp API.

%---------- Methodologie ------------------------------------------------------

\section{Methodologie}%
\label{sec:methodologie}

Het onderzoek wordt opgedeeld in de volgende fasen:

\subsection{Fase 1: Literatuurstudie (2 weken)}

\textbf{Doel}: Diepgaand onderzoek naar de state-of-the-art in conversationele AI en telefonie-integratie.

\textbf{Aanpak}: Analyse van documentatie van Twilio, OpenAI en vergelijkbare platformen. Evaluatie van best practices voor prompt engineering en function calling.

\textbf{Deliverable}: Rapport met technologiekeuze en architectuurvoorstel.

\subsection{Fase 2: Proof of Concept (3 weken)}

\textbf{Doel}: Valideren dat Twilio en OpenAI kunnen samenwerken voor een basis spraakassistent.

\textbf{Aanpak}:
\begin{itemize}
    \item Opzetten Twilio-account met Belgisch telefoonnummer
    \item Implementeren ASP.NET Core backend met webhooks
    \item Integreren OpenAI voor eenvoudige vraag-antwoord flow
\end{itemize}

\textbf{Deliverable}: Werkend prototype dat oproepen kan beantwoorden.

\subsection{Fase 3: Implementatie kernfunctionaliteiten (4 weken)}

\textbf{Doel}: Ontwikkelen van de vijf kernfuncties via function calling.

\textbf{Aanpak}:
\begin{itemize}
    \item Definiëren function calling schema's
    \item Implementeren tool handlers in C\#
    \item Koppelen aan TurnUp API
    \item Iteratief testen en verfijnen van prompts
\end{itemize}

\textbf{Deliverable}: Volledig functioneel systeem met alle kernfuncties.

\subsection{Fase 4: Meertalige ondersteuning (2 weken)}

\textbf{Doel}: Implementeren van Nederlands, Engels en Frans.

\textbf{Aanpak}: IVR-menu voor taalkeuze, taalspecifieke prompts, configuratie spraakherkenning per taal.

\textbf{Deliverable}: Meertalig systeem met passende Text-to-Speech stemmen.

\subsection{Fase 5: Testing en documentatie (3 weken)}

\textbf{Doel}: Uitgebreid testen en documenteren van het systeem.

\textbf{Aanpak}: Handmatige tests, analyse van edge cases, schrijven van technische documentatie en graduaatsproefverslag.

\textbf{Deliverable}: Getest systeem en afgewerkt verslag.

\subsection{Gebruikte tools}

\begin{itemize}
    \item \textbf{Programmeertaal}: C\# (.NET 8)
    \item \textbf{Framework}: ASP.NET Core
    \item \textbf{Telefonie}: Twilio Programmable Voice
    \item \textbf{AI}: OpenAI GPT-4 met function calling
    \item \textbf{Hosting}: Microsoft Azure
    \item \textbf{IDE}: Visual Studio Code, JetBrains Rider
\end{itemize}

%---------- Verwachte resultaten ----------------------------------------------

\section{Verwacht resultaat, conclusie}%
\label{sec:verwachte_resultaten}

\subsection{Verwachte resultaten}

Op basis van de literatuurstudie en vooronderzoek worden de volgende resultaten verwacht:

\begin{itemize}
    \item Een werkend AI-callcenter dat 24/7 beschikbaar is
    \item Succesvolle afhandeling van minimaal 80\% van de standaard scenario's (afspraak opzoeken, annuleren, verzetten)
    \item Gemiddelde gespreksduur van minder dan 3 minuten
    \item Latency van minder dan 5 seconden tussen vraag en antwoord
    \item Volledige meertalige ondersteuning (NL/EN/FR)
\end{itemize}

\subsection{Meerwaarde voor de doelgroep}

Dit onderzoek levert concrete meerwaarde voor TurnUp en hun klanten:

\begin{itemize}
    \item \textbf{Verbeterde bereikbaarheid}: Patiënten kunnen 24/7 terecht, ook buiten kantooruren.
    \item \textbf{Lagere werkdruk}: Receptionisten worden ontlast van repetitieve taken.
    \item \textbf{Kostenefficiëntie}: Geen extra personeel nodig voor telefonische ondersteuning.
    \item \textbf{Schaalbaarheid}: Het systeem kan meerdere gesprekken gelijktijdig afhandelen.
    \item \textbf{Innovatie}: TurnUp kan zich onderscheiden met een moderne AI-oplossing.
\end{itemize}

\subsection{Mogelijke uitbreidingen}

Als het basisproject succesvol is, zijn de volgende uitbreidingen mogelijk:
\begin{itemize}
    \item Outbound calling voor proactieve afspraakherinneringen
    \item Integratie met andere zorgsectoren (huisartsen, kinesisten)
    \item RAG-integratie voor beantwoorden van praktijkspecifieke vragen
\end{itemize}

\subsection{Conclusie}

De combinatie van Twilio voor telefonie en OpenAI voor natuurlijke taalverwerking biedt een veelbelovende basis voor het automatiseren van klantenservice in de zorgsector. Dit onderzoek zal aantonen of en hoe deze technologieën effectief kunnen worden ingezet voor tandartspraktijken, met concrete aanbevelingen voor verdere implementatie.


%% Wijzig "Bibliografie" naar "Bronnenlijst"
\DefineBibliographyStrings{dutch}{
    bibliography = {Bronnenlijst},
}

\printbibliography[heading=bibintoc]

\end{document}
