%%=============================================================================
%% Inleiding
%%=============================================================================

\chapter{\IfLanguageName{dutch}{Inleiding}{Introduction}}%
\label{ch:inleiding}

Tandartspraktijken ontvangen dagelijks tientallen telefoontjes van patiënten voor afspraakgerelateerde vragen. Deze repetitieve taken leggen druk op het administratief personeel en beperken de bereikbaarheid tot kantooruren. TurnUp, een Belgisch softwarebedrijf gespecialiseerd in praktijkbeheersoftware, zag een opportuniteit om deze uitdaging aan te pakken met AI-technologie.

\section{Probleemstelling}%
\label{sec:probleemstelling}

Tandartspraktijken worden geconfronteerd met:
\begin{itemize}
    \item \textbf{Beperkte bereikbaarheid}: Enkel tijdens kantooruren
    \item \textbf{Wachttijden}: Piekmomenten leiden tot frustratie
    \item \textbf{Repetitieve taken}: Standaardvragen die geautomatiseerd kunnen worden
    \item \textbf{Meertaligheid}: NL/FR/EN ondersteuning vereist
\end{itemize}

\section{Onderzoeksvraag}%
\label{sec:onderzoeksvraag}

\textit{Hoe kan een AI-gestuurd callcenter de telefonische dienstverlening van tandartspraktijken automatiseren met behoud van klanttevredenheid?}

Deelvragen:
\begin{enumerate}
    \item Welke technologieën zijn geschikt voor spraakgestuurde AI-assistenten?
    \item Hoe kan OpenAI's function calling worden ingezet voor afspraakbeheer?
    \item Welke architectuur is nodig voor real-time spraakverwerking?
    \item Hoe kan meertalige ondersteuning worden geïmplementeerd?
\end{enumerate}

\section{Onderzoeksdoelstelling}%
\label{sec:onderzoeksdoelstelling}

Het beoogde resultaat is een werkend prototype dat:
\begin{itemize}
    \item Inkomende telefoontjes automatisch beantwoordt via Twilio
    \item Patiënten in natuurlijke taal te woord staat via OpenAI GPT
    \item Ondersteuning biedt voor Nederlands, Engels en Frans
    \item Afspraken kan opzoeken, annuleren, verzetten en patiënten aan de wachtlijst kan toevoegen
    \item Integreert met de TurnUp API
\end{itemize}
